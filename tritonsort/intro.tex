\section{Introduction}
\label{tritonsort:sec:intro}

In this work we present \tritonsort, a highly efficient sorting system designed
to sort large volumes of data across dozens of nodes. We have applied it to
data sets as large as 100 terabytes spread across \tsdisks disks in \tsnodes
nodes.  The key to \tritonsort's efficiency is its \textit{balanced} software
architecture, which is able to effectively make use of a large amount of
co-located storage per node, ensuring that the disks are kept as utilized as
possible.  Our results show the benefit of our design: evaluating \tritonsort
against the `Indy' GraySort benchmark\cite{terasort} resulted in a system that
was able to sort 100TB of input records in about \tspercent of the absolute time
of the previous record-holder, but with four times fewer resources, resulting
in an increase in per-node efficiency by over a factor of \tsimprovementfactor.

It is important to note that our focus in building \tritonsort is to highlight
the efficiency gains that can be obtained in building systems that process
significant amounts of data through balancing computation, storage, memory, and
network.  Systems such as Hadoop and Dryad further support data-level
replication, transparent node failure, and a generalized computational model,
all of which are not currently present in \tritonsort.  However, in presenting
\tritonsort's hardware and software architecture, we describe several lessons
learned in its construction that we believe are generalizable to other data
processing systems.  For example, our design relies on a very high disk-to-node
ratio as well as an explicit, application-level management of in-memory buffers
to minimize disk seeks and thus increase read and write throughput.  We choose
buffer sizes to balance time spent processing multiple stages of our sort
pipeline, and trade off the utilization of one resource for another.

Our experiences show that for a common datacenter workload, systems can be
built with commodity hardware and open-source software that improve on per-node
efficiency by an order of magnitude while still achieving scalability.
Building such systems will either enable significantly cheaper systems to be
able to do the same work or provide the ability to address significantly larger
problem sets with the same infrastructure.

% LocalWords:  Google Mbps GraySort records pipelined LocalWords
