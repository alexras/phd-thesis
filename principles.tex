\chapter{Architectural and Design Principles}
\label{chapter:principles}

\section{Building a ``Balanced'' System}

Both TritonSort and Themis aim to ensure good resource utilization by being
``balanced'' systems. We define a balanced system as one that drives all
cluster resources at as close to 100\% utilization as possible. For any given
application and workload, there will be an ideal hardware configuration in
keeping with the application's demands on a cluster's resources. In practice,
however, the set of hardware configurations is limited by the availability of
components; for example, one cannot currently buy a processor with precisely 13
cores. As a result, a hardware configuration must be chosen that best meets the
application's demands.  One the appropriate hardware configuration is
determined, the application must be architected to suitably exploit the
hardware's capabilities. In the following sections, we outline our
considerations in designing a balanced system, including our choices of
hardware and software architecture.

\section{Design Considerations}

Our system's design is motivated by three main considerations.  First, we rely
only on commodity hardware components.  This is both to keep the costs of our
system relatively low and to have our system be representative of today's data
centers so that the lessons we learn can be applied to other applications with
similar workload characteristics.  Hence, we do not make use of networking
substrates like Infiniband that provide high network bandwidth at high cost.
Also, despite the recent emergence of solid state drives (SSDs) that provide
higher I/O rates, we chose to use hard disks because they continue to provide
the most affordable option for high capacity storage and streaming I/O.  We
believe that properly-architected data-intensive software should not stress
random I/O behavior, where SSDs currently excel.

Second, we focus our software architecture on minimizing both disk seeks and
disk I/O.  In the particular hardware configuration we chose, the key
bottleneck among the various system resources is disk I/O bandwidth.  Hence,
the primary goal of the system is to enable all disks to operate continuously
at peak bandwidth.  The main challenge in sustaining peak disk bandwidth is to
minimize the amount of time the disks spend seeking, since any time
seeking is not spent transferring data

Additionally, we seek to minimize the number of times each record is
transferred from disk.  Sorting data on clusters that have less memory than the
total amount of data to be sorted requires every input record to be read and
written at least twice~\cite{sort-io}.  Since a distributed sort by key is the
kernel of any MapReduce job, this lower-bound also applies to MapReduce. Since
every additional read and write increases the time to sort, we seek to achieve
exactly this lower bound to maximize system performance.

Third, we choose to focus on hardware architectures whose total memory cannot
contain the entire dataset, because such a design would significantly drive up
costs and be infeasible for input datasets at the scales that we consider in
this dissertation. Significant improvements in I/O-efficiency are possible when
the dataset fits in memory; we explore these implications briefly in
Chapter~\ref{chapter:tritonsort}.

\section{Hardware Architecture}

To determine the right hardware configuration for our application, we make the
following observations about our workloads. Since the ``working set'' for our
data is so large, it does not make sense to separate the cluster into
computation-heavy and storage-heavy regions. Instead, we provision each server
in the cluster with an equal amount of processing power and disks.

\begin{table}[t]
\caption{Resource options considered for constructing a cluster for a
  balanced sorting system.  These values are estimates as of January, 2010.}
\label{tab:resourcesummary}
\begin{center}
\begin{tabular}{|c|c|c|c|}
\hline
\multicolumn{4}{|c|}{{\bf Storage}}\\
\hline
Type & Capacity & R/W throughput & Price\\
\hline
7.2k-RPM & 500 GB & 90-100 MBps & \$200\\
\hline
15k-RPM & 150 GB & 150 MBps & \$290\\
\hline
SSD     & 64 GB  & 250 MBps & \$450\\
\hline
\multicolumn{4}{c}{}\\
\hline
\multicolumn{4}{|c|}{{\bf Network}}\\
\hline
\multicolumn{3}{|c|}{Type} & Cost/port\\
\hline
\multicolumn{3}{|c|}{1 Gbps Ethernet} & \$33\\
\hline
\multicolumn{3}{|c|}{10 Gbps Ethernet} & \$480\\
%\hline
%\multicolumn{3}{|c|}{40 Gbps Ethernet} & \$ZZZ\\
\hline
\multicolumn{4}{c}{}\\
\hline
\multicolumn{4}{|c|}{{\bf Server}}\\
\hline
\multicolumn{3}{|c|}{Type} & Cost\\
\hline
\multicolumn{3}{|c|}{8 disks, 8 CPU cores} & \$5,050\\
\hline
\multicolumn{3}{|c|}{8 disks, 16 CPU cores} & \$5,450\\
\hline
\multicolumn{3}{|c|}{16 disks, 16 CPU cores} & \$7,550\\
\hline
\end{tabular}
\end{center}
\end{table}

Second, almost all of the data needs to be exchanged between machines as part
of the shuffle step of the computation.  To balance the system, we need to
ensure that this all-to-all shuffling of data can happen in parallel without
network bandwidth becoming a bottleneck.  Since we focus on using commodity
components, we use an Ethernet network fabric.  Commodity Ethernet is available
in a set of discrete bandwidth levels---1 Gbps, 10 Gbps, and 40 Gbps---with
cost increasing proportional to throughput (see
Table~\ref{tab:resourcesummary}).  Given our choice of 7.2k-RPM disks for
storage, a 1 Gbps network can accommodate at most one disk per server without
the network throttling disk I/O.  Therefore, we settle on a 10 Gbps network; 40
Gbps Ethernet has yet to mature and hence is still cost-prohibitive.  To
balance a 10 Gbps network with disk I/O, we use a server that can host 16
disks.  Based on the options available commercially for such a server, we use a
server that hosts 16 disks and 8 CPU cores.  The choice of 8 cores was driven
by the available processor packaging: two physical quad-core CPUs.  The larger
the number of separate threads, the more stages that can be isolated from each
other.  In our experience, the actual speed of each of these cores was a
secondary consideration, since the workloads we consider are mostly heavily
I/O-bound.

Third, our problem domains require both significant capacity and I/O
requirements from storage, since tens to hundreds of TB of data is to be stored
and all the data is to be read and written twice.  To determine the best
storage option given these requirements, we survey a range of hard disk options
shown in Table~\ref{tab:resourcesummary}.  We find that 7.2k-RPM SATA disks
provide the most cost-effective option in terms of balancing cost per GB and
cost per read/write MBps (assuming we can achieve streaming I/O).  To allow 16
disks to operate at full streaming I/O throughput, we require storage
controllers that are able to sustain at least 1600 MBps of streaming bandwidth.
Our hardware design necessitated two 8x PCI drive controllers, each supporting
8 disks, because of the PCI bus' bandwidth limitations.

The final design choice in provisioning our cluster is the amount of memory
each server should have.  The primary purpose of memory in our system is to
enable large amounts of data buffering so that we can read from and write to
the disk in large chunks.  The larger these chunks become, the more data can be
read or written before seeking is required.  We initially provisioned each of
our machines with 12 GB of memory; however, during development we realized that
24 GB was required to provide sufficiently large writes, and so the machines
were upgraded.  One of the key takeaways from our work is the important role
that buffering plays in enabling high utilization of the network, disk, and
CPU, and the efficient, dynamic management of that buffering is a key
contribution of this work.

\section{Software Architecture}

TritonSort and Themis are staged, pipeline-oriented dataflow processing
systems. Both systems are implemented as directed graphs of \emph{stages}. Each
stage implements part of the data processing pipeline and either sources,
sinks, or transmutes data flowing through it.

Each stage is implemented by a collection of \emph{workers}, each of which is a
separate thread. Workers receive input \emph{work units}, which are typically
in-memory buffers, by dequeuing them from a collection of per-stage queues. In
the process of running, a worker can produce work for a downstream stage, and
optionally direct the worker to which that work should be directed. If a worker
does not specify a destination worker, work units are assigned according to a
per-stage work queueing policy that defaults to round-robin. All workers in a
given stage graph run in parallel.

When a work unit arrives, the worker executes a stage-specific \texttt{run()}
method that implements the specific function of the stage. Workers process work
in one of three ways. First, a worker can accept an individual work unit,
execute the \texttt{run()} method over it, and then wait for new work. Second,
it can accept a batch of work (up to a configurable size) that has been
enqueued to one of its stage's queues. Lastly, it can keep its \texttt{run()}
method active, polling for the presence of new work units
explicitly. TritonSort and Themis contain examples of each of these three kinds
of methods.
