\documentclass[12pt]{externals/ucsddissertation/ucsddissertation}
% mathptmx is a Times Roman look-alike (don't use the times package)
% It isn't clear if Times is required. The OGS manual lists several
% "standard fonts" but never says they need to be used.
\usepackage{mathptmx}
\usepackage[NoDate]{currvita}
\usepackage{array}
\usepackage{tabularx}
\usepackage{booktabs}
\usepackage{ragged2e}
\usepackage{microtype}
\usepackage[breaklinks=true,pdfborder={0 0 0}]{hyperref}
\usepackage{graphicx}
\AtBeginDocument{%
	\settowidth\cvlabelwidth{\cvlabelfont 0000--0000}%
}

% OGS recommends increasing the margins slightly.
\increasemargins{.1in}

% These are just for testing/examples, delete them
\usepackage{trace}
%\usepackage{showframe} % This package was just to see page margins
\usepackage[english]{babel}
\usepackage{blindtext}
\overfullrule5pt

\def\tritonsort{TritonSort\xspace}
\def\themis{Themis\xspace}

% -- STAGE NAME DEFINITIONS --

\def\Reader{Reader\xspace}
\def\Readerbf{\textbf{Reader}\xspace}
\def\reader{reader\xspace}
\def\readers{readers\xspace}

\def\Mapper{Mapper\xspace}
\def\Mapperbf{\textbf{Mapper}\xspace}
\def\mapper{mapper\xspace}
\def\mappers{mappers\xspace}

\def\Sender{Sender\xspace}
\def\Senderbf{\textbf{Sender}\xspace}
\def\sender{sender\xspace}
\def\senders{senders\xspace}

\def\Receiver{Receiver\xspace}
\def\Receiverbf{\textbf{Receiver}\xspace}
\def\receiver{receiver\xspace}
\def\receivers{receivers\xspace}

\def\Demux{Demux\xspace}
\def\Demuxbf{\textbf{Demux}\xspace}
\def\demux{demux\xspace}
\def\demuxes{demuxes\xspace}

\def\Chainer{Chainer\xspace}
\def\Chainerbf{\textbf{Chainer}\xspace}
\def\chainer{chainer\xspace}
\def\chainers{chainers\xspace}

\def\Coalescer{Coalescer\xspace}
\def\Coalescerbf{\textbf{Coalescer}\xspace}
\def\coalescer{coalescer\xspace}
\def\coalescers{coalescers\xspace}

\def\Writer{Writer\xspace}
\def\Writerbf{\textbf{Writer}\xspace}
\def\writer{writer\xspace}
\def\writers{writers\xspace}

\def\ByteStreamConverter{Byte Stream Converter\xspace}
\def\ByteStreamConverterbf{\textbf{Byte Stream Converter}\xspace}
\def\bytestreamconverter{byte stream converter\xspace}
\def\bytestreamconverters{byte stream converters\xspace}

\def\Sorter{Sorter\xspace}
\def\Sorterbf{\textbf{Sorter}\xspace}
\def\sorter{sorter\xspace}
\def\sorters{sorters\xspace}

\def\Reducer{Reducer\xspace}
\def\Reducerbf{\textbf{Reducer}\xspace}
\def\reducer{reducer\xspace}
\def\reducers{reducers\xspace}

% -- END STAGE NAME DEFINITIONS --

\def\map{\texttt{map}\xspace}
\def\reduce{\texttt{reduce}\xspace}


% ---

% Required information
\title{I/O-Efficient Data-Intensive Processing}
\author{Alexander Rasmussen}
\degree{Computer Science and Engineering}{Doctor of Philosophy}
% Each member of the committee should be listed as Professor Foo Bar.
% If Professor is not the correct title for one, then titles should be
% omitted entirely.
\chair{Professor Amin Vahdat}
% Your committee members (other than the chairs) must be in alphabetical order
\committee{Professor Alin Deutsch}
\committee{Professor Tara Javidi}
\committee{Professor Bill Lin}
\committee{Professor Geoffrey Voelker}
\degreeyear{2013}

\newcommand{\kvpair}[2]{{\tt <#1, #2>}}

% Start the document
\begin{document}
% Begin with frontmatter and so forth
\frontmatter
\maketitle
\makecopyright
\makesignature

\begin{epigraph}
\vskip0pt plus.5fil
\setsinglespacing
{\flushright
Speed provides the one genuinely modern pleasure.
\vskip\baselineskip
\textit{Aldous Huxley}\par}
\vfil
{\flushright
Obviously, the highest type of efficiency is that which \\can utilize existing material to the best advantage.
\vskip\baselineskip
\textit{Jawaharlal Nehru}\par}
\vfil
{\flushright
Efficiency is doing things right; effectiveness is doing the right things.
\vskip\baselineskip
\textit{Peter Drucker}\par}
\vfil
\end{epigraph}


% Next comes the table of contents, list of figures, list of tables,
% etc. If you have code listings, you can use \listoflistings (or
% \lstlistoflistings) to have it be produced here as well. Same with
% \listofalgorithms.
\tableofcontents
\listoffigures
\listoftables

% Your fancy acks here. Keep in mind you need to ack each paper you
% use. See the examples here. In addition, each chapter ack needs to
% be repeated at the end of the relevant chapter.
\begin{acknowledgements}

First and foremost, thanks to my parents. Throughout the past five years, being
close to home was a great source of comfort. You believed I could do this, even
when I didn't believe it myself. Thank you.

I would like to acknowledge Professor Amin Vahdat for his support as the chair
of my committee.  His support and patience throughout this process have been
invaluable.

I would also like to acknowledge George Porter, whose herculean efforts in
co-authorship, cluster management, grant writing and advising have helped
both myself and many others beyond measure.

This dissertation would not have been possible without the hard work and
support of my co-authors. Thank you all.

Many thanks to my friends, both inside and outside UCSD, who gave me their
unwavering support (and, in some cases, the use of their couch). I am
profoundly blessed to have all of you in my life.

Parts of this dissertation are based on papers co-authored with my
collaborators:

\begin{itemize}
  \item Chapter~\ref{chapter:tritonsort} is based on ``TritonSort: A Balanced
    Large-Scale Sorting System'', by Alexander Rasmussen, George Porter,
    Michael Conley, Harsha Madhyastha, Radhika Mysore, Alexander Pucher, and
    Amin Vahdat, as it appears in~\cite{tritonsort}.
  \item Chapter~\ref{chapter:themis} is based on ``Themis: An I/O Efficient
    MapReduce'', by Alexander Rasmussen, Michael Conley, Rishi Kapoor, Vinh The
    Lam, George Porter, and Amin Vahdat, as it appears in ~\cite{themis}.
  \item Chapter~\ref{chapter:fault_tolerance} in part is currently being
    prepared for submission for publication of the material by Alexander
    Rasmussen, George Porter and Amin Vahdat.
\end{itemize}

\end{acknowledgements}


% Stupid vita goes next
\begin{vita}
\noindent
\begin{cv}{}
\begin{cvlist}{}
\item[2007] Bachelor of Science, University of California, Berkeley
\item[2010] Masters of Science, University of California, San Diego
\item[2013] Doctor of Philosophy, University of California, San Diego
\end{cvlist}
\end{cv}

\publications
\noindent``Themis: An I/O-Efficient MapReduce'' ACM Symposium on Cloud
Computing, October 2012.

\noindent``TritonSort: A Balanced Large-Scale Sorting System'' USENIX Symposium
on Networked Systems Design and Implementation, April 2011.

\noindent``Short Paper: Improving the Responsiveness of Internet Services with
Automatic Cache Placement'' European Conference in Computer Systems, April 2009.

\end{vita}


% Put your maximum 350 word abstract here.
\begin{dissertationabstract}
There is a growing need for scalable, data-intensive processing platforms to
analyze and filter the large volume of data collected by both corporations and
scientific institutions. The effectiveness of these systems is measured by the
quantity and quality of data that they can process in a reasonable amount of
time; thus, these systems have very high I/O and storage requirements.

Existing systems are very effective at scaling to large cluster
sizes. Unfortunately, there exists a significant gap between the performance
these systems provide and the underlying capacity of the hardware
infrastructure on which they are deployed.

In this dissertation, I endeavor to bridge this performance gap by focusing on
efficient I/O as a first-class architectural concern. In particular, I present
two systems, TritonSort and Themis. TritonSort is a high-performance
large-scale sorting system capable of sorting 100TB of data on a modestly-sized
cluster at about 82\% of that cluster's peak hardware performance. Themis is a
successor system to TritonSort that supports the popular MapReduce programming
paradigm and can run a wide spectrum of MapReduce jobs at nearly the speed at
which TritonSort can sort. I conclude with the implementation of a fault
tolerance scheme for Themis that provides proportional fault tolerance without
imposing additional rounds of I/O during common-case operation.
\end{dissertationabstract}


% This is where the main body of your dissertation goes!
\mainmatter
\chapter{Introduction}
\label{chapter:introduction}

The quantity of data the world generates and stores is growing at a staggering
rate. Online retailers log users' purchase histories and interactions with
their websites in order to target advertising to their particular
interests. Walmart handles more than a million customer transactions per hour,
and the size of its customer database is estimated at 2.5
petabytes~\cite{economist-data-data-everywhere}. Search engines like Google
construct complex indices over the entire public Internet, which is estimated
to consist of at least 6.9 billion pages~\cite{worldwidewebsize}. Facebook's
users upload more than 300 million photos per
day~\cite{jay-parikh-slideshow}. Scientific instruments like the Australian
Square Kilometer Array, the Large Hadron Collider and the Pan-STARRS array of
telescopes can generate petabytes of data per day~\cite{fourth-paradigm}.

Capturing this data, while a technically challenging feat in and of itself, is
not enough. To be useful, this large volume of data must be analyzed,
aggregated, filtered, transformed and explored. This is no easy task. The
aforementioned data sets are but a few examples of a new class of ``big data''
-- data sets that are so large and complex that they become difficult to
process using traditional techniques and technologies.

Some data-intensive problems allow every record to be processed in parallel
without knowing anything about any other records. These problems are known as
``embarrassingly parallel'', and can be scaled out quite easily. For this class
of problems, simply splitting the data into small pieces and running the
desired computation over each piece is sufficient and can be scaled quite
easily. An example of an embarrassingly parallel problem is searching a corpus
of text for occurrences of a word; each document in the corpus can be scanned
independently and the results of the scan over each document can be trivially
merged together afterward.

However, a much larger class of problems are not embarrassingly parallel. These
problems require some form of aggregation or combination (analogous to group-by
and join in relational database systems) in addition to record-by-record
processing. Take, for example, the problem of counting the number of times each
word occurs in a corpus of text. The occurrences of each word can be counted in
each document independently, but these counts must then be added
together. Performing this aggregation efficiently is one of the primary
challenges facing designers of systems for computing on ``big data''.

\section{The Rise of Scale-Out Partition-Parallel Architectures}


In recent years, a range of large-scale, data-intensive systems have been
developed to tackle these kinds of workloads.  These systems work by
performing both record-by-record processing and aggregation in parallel whenever
possible by performing the computation over disjoint partitions of the
data.

One of the most popular frameworks for this form of analysis is
MapReduce~\cite{mapreduce}. A MapReduce computation is specified by two
functions. The first function, \map, takes a record as input and produces
zero or more records; it performs the record-by-record processing. Map
functions are generally assumed to be stateless and side-effect free so that
they are idempotent. The second function, \reduce, takes all records with
the same key as input and produces zero or more records. Colloquially, records
that have been passed through the \map function are said to have been
\emph{mapped}, while those that have been passed through the \reduce function
are said to have been \emph{reduced}.

MapReduce treats a data set as a collection of \emph{records}, each of which
consists of a \emph{key} and a \emph{value}. Both the key and the value can be
arbitrary. A record with key $k$ and value $v$ will be denoted \kvpair{k}{v}.

A canonical example MapReduce job is the problem of counting the occurrences of
each word in a text corpus. For this problem, the user might write a \map
function that takes a line of text as input and produces the record
\kvpair{word}{1} for each word in the line. The \reduce function would then
receive all records for a given word, add their values together, and produce a
single record \kvpair{word}{n}.

MapReduce's strength lies in the simplicity of its programming model. Users of
MapReduce need only write map and \reduce functions without concerning
themselves with dividing the data among nodes, performing inter-node
communication or handling node failure. The \map function's domain and its
idempotent nature make it embarrassingly parallel, while the \reduce function's
parallelism can be adjusted from completely serial to extremely parallel based
on the number of distinct keys in the records produced by the \map function.

MapReduce was developed by Google in the early 2000s for tasks like inverted
index generation and PageRank~\cite{pagerank} computation over Google's cache
of the web. Engineers at Yahoo! wrote an open-source version of MapReduce
called Hadoop~\cite{hadoop} in 2005 that has since become extremely popular and
is widely deployed in both academic and industrial settings.

\section{Scale-Out, but not Scale-Up}

While these systems scale quite well, they do not utilize their clusters'
resources to nearly the extent that they should. As one example, a cluster of
3452 nodes running Hadoop sorted 100 TB of data in 173
minutes~\cite{hadoop-sort-2009}. At a high level, this performance is quite
impressive -- an average of 578 GB of data sorted per minute. However, this
high-level performance masks a great deal of inefficiency. In the
aforementioned record-setting sort, each node in the cluster's average rate was
approximately 2.8 MBps, a small fraction of the speed at which that can read
from and write to its disks. This performance difference is even more shocking
when one considers that a significant fraction of the data (approximately 27
TB) could conceivably have been buffered in the cluster's main memory for
faster access.

These problems are not limited to Hadoop, however. Anderson and
Tucek~\cite{efficiency-matters} examined a collection of large-scale
data-intensive processing systems and found a widespread lack of efficiency
among them.

\begin{table}
\caption{\label{table:system-efficiency} Large scale sorting results over time,
  and their associated per-node and per-disk efficiency. Results extracted from
  ~\cite{efficiency-matters, hadoop-sort-2009, themis, tritonsort}.}
\centering
\begin{tabular}{|c|c|c|c|c|c|c|c|}
\hline
\textbf{Year} & \textbf{Name} & \textbf{Nodes} & \textbf{Disks} & \textbf{MB/s} & \textbf{MB/s/node} & \textbf{MB/s/disk}\\
\hline
2012 & Themis (35TB) & 20 & 320 & 4656 & 232.8 & 14.6 \\
2011 & Themis & 52 & 832 & 12080 & 232.4 & 14.5 \\
2011 & TritonSort & 52 & 832  & 15633 & 300.6 & 18.8 \\
2009 & Hadoop & 3452 & 13808 & 9633 & 2.79 & 0.69 \\
2009 & DEMSort & 195 & 780 & 9400 & 48.2 & 12.1\\
\hline
\end{tabular}
\end{table}

The tempting solution to the problem of low efficiency is to simply increase
the size of the cluster. Splitting the data being processed among progressively
more nodes decreases both the amount of data that each node must process and
(to the extent allowed by Amdahl's Law~\cite{amdahls_law}) increases the
throughput of the system. However, this approach has several negative
consequences.

Larger clusters have a proportionally large associated capital expense and
operational cost. Google, one of the pioneers of the current wave of largest
data-intensive systems, has contracted over 260MW to power its data
centers~\cite{google-dc-power-blog}. When it filed its IPO in 2011, Facebook
reported that it spent \$606 million on constructing and equipping its data
centers in 2011 and expected to spend another \$500 million in
2012~\cite{facebook-ipo}. As problem sizes increase, these expenses must also
increase.

Large data centers also have an environmental cost. McKinsey and Company
estimates that the carbon dioxide emissions from data centers will surpass
emissions from the airline industry by 2020~\cite{mckinsey-co2-study}.
Further, larger clusters are harder to manage and experience faults more
frequently than smaller clusters do because of the increased number of nodes in
those clusters. We will explore the implications of increased failure further
in Chapters~\ref{chapter:themis} and~\ref{chapter:fault_tolerance}.

\section{Sources of Inefficiency in Existing Systems}

While a thorough study of the sources of per-node inefficiency in existing
systems has not been performed, we can broadly classify three different sources
of inefficiency in systems that are I/O-bound:

\paragraph{Inefficient I/O} Current-generatione large scale data processing
systems read from and write to large collections of magnetic hard drives. These
magnetic drives are characterized by their fast sequential access and slow
random access. Fundamentally, systems that desire a high throughput from these
devices should write to them sequentially as much as possible.

\paragraph{Too much I/O per record} Existing
systems may read and write each record to disk several times before processing
is complete, either due to memory pressure or for fault tolerance. These
additional reads and writes incur significant additional overhead, as disk is
at least an order of magnitude slower than main memory or network transfer.

\paragraph{Imbalanced hardware configurations} Often, the hardware platforms on
which these systems are deployed are configured such that the system will run
out of network bandwidth or memory before they can maximize their disks'
throughput. This dissertation argues that a degree of software/hardware
co-design can lead to radically more efficient software and hardware
architectures.

\section{Hypothesis}

The hypothesis of this dissertation is that systems built with efficient disk
I/O as a first-order architectural concern can realize an order of magnitude
improvement in performance versus existing large-scale data-intensive systems
without compromising their scalability or generality.

In this dissertation, we argue that the chief challenges of building such a
system lie both in minimizing the number of I/O operations per record and in
ensuring that disk I/O is done sequentially as much as possible. We also argue
that significant increases in per-node efficiency can be realized by
considering alternative fault tolerance models to the task-level fault
tolerance schemes present in modern data-intensive systems.

We explore the design of radically more efficient data processing systems
through two main prototype systems: TritonSort, a large-scale sorting system,
and Themis, a large-scale MapReduce system.

\section{Organization}

TritonSort and Themis have each improved on the performance of systems in their
respective problem domains by almost an order of magnitude, approaching the
maximum throughput possible on the clusters on which they are
deployed. Further, we have demonstrated both analytically and experimentally
that a range of different fault tolerance schemes that incur less I/O than the
traditional task-based approach to fault tolerance are practical for a
wide-range of MapReduce deployments, and implemented a fault tolerance scheme
that does not impose additional rounds of disk I/O during failure-free
operation.

Chapter~\ref{chapter:background} provides background on the problem domains of
large-scale sorting and MapReduce. Chapter~\ref{chapter:principles} provides an
overview of the architecture and design principles that underpin the systems
presented in this dissertation. Chapter~\ref{chapter:tritonsort} presents the
design and implementation of TritonSort. Chapter~\ref{chapter:themis} presents
the design and implementation of Themis, focusing in particular on its
differences from TritonSort's design. Chapter~\ref{chapter:fault_tolerance}
takes an in-depth look at fault tolerance in Themis, looking first at the
trade-off between fault tolerance and I/O-efficiency, and then presenting the
design and implementation of an I/O-efficient fault tolerance scheme for
Themis. Chapter~\ref{chapter:related} explores related work.  The dissertation
concludes with Chapter~\ref{chapter:conclusions}, which describes some open
problems and future directions.

\chapter{Background}
\label{chapter:background}

This section makes the problem domains tackled by TritonSort and Themis more
concrete, and describes the architectural features that both systems share in
common.

\section{Problem Formulation: Sorting}

TritonSort seeks to meet the specifications laid out in the GraySort
benchmark~\cite{terasort}. For this benchmark, the data to be sorted consists of
100 byte records, each of which has a 10-byte key and a 90-byte value. We
target deployments with input datasets that are tens to hundreds of terabytes
in size; the GraySort benchmark's current data size is 100 terabytes.

Input data is stored as a collection of files across the cluster's
disks. TritonSort's goal is to transform this input data set into an ordered
set of output files, also stored across the cluster's disks, such that an
in-order concatenation of these output files is a sorted permutation of the
input data set.

Sorting large datasets places great stress on a cluster's resources.  First,
storing tens to hundreds of terabytes of data demands a large amount of storage
capacity. Given the capacity of modern hard drives, the data must be stored
across several drives and almost certainly across many machines. Second,
performing reads and writes to all these disks simultaneously places load on
both the disks themselves and the I/O controllers connecting them to the
CPU. Third, since the records to be sorted are assumed to be distributed
randomly across input files, almost all of the dataset to be sorted will have
to be sent over the network at some point. Finally, comparing records requires
a non-trivial amount of compute power. This combination of demands makes it
challenging to design an efficient large-scale sorting system that utilizes the
cluster's resources well.

\section{Problem Formulation: MapReduce}

As mentioned in Chapter~\ref{chapter:introduction}, a MapReduce computation is
specified by two functions, \map and \reduce, with \map responsible for
per-record processing and \reduce responsible for aggregation. MapReduce treats
a data set as a collection of \emph{records}, each of which consists of a
\emph{key} and a \emph{value}. Both the key and the value can be arbitrary. A
record with key $k$ and value $v$ will be denoted \kvpair{k}{v}.  Throughout
this dissertation we will refer to records that are produced by the \map
function as \emph{intermediate records} or \emph{mapped records} and records
produced by the \reduce function as \emph{output records} or \emph{reduced
  records}.

A canonical example MapReduce job is the problem of counting the occurrences of
each word in a text corpus. For this problem, the user might write a \map
function that takes a line of text as input and produces the record
\kvpair{word}{1} for each word in the line. The \reduce function would then
receive all records for a given word, add their values together, and produce a
single record \kvpair{word}{n}.

The \map and \reduce functions can produce an arbitrary number of
arbitrarily-sized records. This is in sharp contrast to the GraySort benchmark,
where records can be assumed to be the same size. Additionally, keys can be
arbitrarily distributed throughout the space of possible keys. This makes the
problem of evenly dividing key ranges among nodes difficult, as we will see in
Chapter~\ref{chapter:themis}.

Since each \reduce function is responsible for processing all records with the
same key, the system running a MapReduce job must ensure that all records with
the same key are available on the same node. This property requires that the
system perform a distributed sort of all intermediate records by key before
applying the \reduce function to each key's records. In this way, the problem
of efficiently running MapReduce jobs is a superset of the problem of
efficiently sorting at scale; in fact, a sort job in MapReduce is simply a job
with ``no-op'' \map and \reduce functions that emit any records they receive
unmodified. As we will see later in this dissertation, we applied many of the
lessons learned in designing an efficient large-scale sorting system to the
problem of building an efficient MapReduce platform.

\chapter{Architectural and Design Principles}
\label{chapter:principles}

\section{Building a ``Balanced'' System}

Both TritonSort and Themis aim to ensure good resource utilization by being
``balanced'' systems. We define a balanced system as one that drives all
cluster resources at as close to 100\% utilization as possible. For any given
application and workload, there will be an ideal hardware configuration in
keeping with the application's demands on a cluster's resources. In practice,
however, the set of hardware configurations is limited by the availability of
components; for example, one cannot currently buy a processor with precisely 13
cores. As a result, a hardware configuration must be chosen that best meets the
application's demands.  One the appropriate hardware configuration is
determined, the application must be architected to suitably exploit the
hardware's capabilities. In the following sections, we outline our
considerations in designing a balanced system, including our choices of
hardware and software architecture.

\section{Design Considerations}

Our system's design is motivated by three main considerations.  First, we rely
only on commodity hardware components.  This is both to keep the costs of our
system relatively low and to have our system be representative of today's data
centers so that the lessons we learn can be applied to other applications with
similar workload characteristics.  Hence, we do not make use of networking
substrates like Infiniband that provide high network bandwidth at high cost.
Also, despite the recent emergence of solid state drives (SSDs) that provide
higher I/O rates, we chose to use hard disks because they continue to provide
the most affordable option for high capacity storage and streaming I/O.  We
believe that properly-architected data-intensive software should not stress
random I/O behavior, where SSDs currently excel.

Second, we focus our software architecture on minimizing both disk seeks and
disk I/O.  In the particular hardware configuration we chose, the key
bottleneck among the various system resources is disk I/O bandwidth.  Hence,
the primary goal of the system is to enable all disks to operate continuously
at peak bandwidth.  The main challenge in sustaining peak disk bandwidth is to
minimize the amount of time the disks spend seeking, since any time
seeking is not spent transferring data

Additionally, we seek to minimize the number of times each record is
transferred from disk.  Sorting data on clusters that have less memory than the
total amount of data to be sorted requires every input record to be read and
written at least twice~\cite{sort-io}.  Since a distributed sort by key is the
kernel of any MapReduce job, this lower-bound also applies to MapReduce. Since
every additional read and write increases the time to sort, we seek to achieve
exactly this lower bound to maximize system performance.

Third, we choose to focus on hardware architectures whose total memory cannot
contain the entire dataset, because such a design would significantly drive up
costs and be infeasible for input datasets at the scales that we consider in
this dissertation. Significant improvements in I/O-efficiency are possible when
the dataset fits in memory; we explore these implications briefly in
Chapter~\ref{chapter:tritonsort}.

\section{Hardware Architecture}
\label{sec:hardware_architecture}

To determine the right hardware configuration for our application, we make the
following observations about our workloads. Since the ``working set'' for our
data is so large, it does not make sense to separate the cluster into
computation-heavy and storage-heavy regions. Instead, we provision each server
in the cluster with an equal amount of processing power and disks.

\begin{table}[t]
\caption{Resource options considered for constructing a cluster for a
  balanced sorting system.  These values are estimates as of January, 2010.}
\label{tab:resourcesummary}
\begin{center}
\begin{tabular}{|c|c|c|c|}
\hline
\multicolumn{4}{|c|}{{\bf Storage}}\\
\hline
Type & Capacity & R/W throughput & Price\\
\hline
7.2k-RPM & 500 GB & 90-100 MBps & \$200\\
\hline
15k-RPM & 150 GB & 150 MBps & \$290\\
\hline
SSD     & 64 GB  & 250 MBps & \$450\\
\hline
\multicolumn{4}{c}{}\\
\hline
\multicolumn{4}{|c|}{{\bf Network}}\\
\hline
\multicolumn{3}{|c|}{Type} & Cost/port\\
\hline
\multicolumn{3}{|c|}{1 Gbps Ethernet} & \$33\\
\hline
\multicolumn{3}{|c|}{10 Gbps Ethernet} & \$480\\
%\hline
%\multicolumn{3}{|c|}{40 Gbps Ethernet} & \$ZZZ\\
\hline
\multicolumn{4}{c}{}\\
\hline
\multicolumn{4}{|c|}{{\bf Server}}\\
\hline
\multicolumn{3}{|c|}{Type} & Cost\\
\hline
\multicolumn{3}{|c|}{8 disks, 8 CPU cores} & \$5,050\\
\hline
\multicolumn{3}{|c|}{8 disks, 16 CPU cores} & \$5,450\\
\hline
\multicolumn{3}{|c|}{16 disks, 16 CPU cores} & \$7,550\\
\hline
\end{tabular}
\end{center}
\end{table}

Second, almost all of the data needs to be exchanged between machines as part
of the shuffle step of the computation.  To balance the system, we need to
ensure that this all-to-all shuffling of data can happen in parallel without
network bandwidth becoming a bottleneck.  Since we focus on using commodity
components, we use an Ethernet network fabric.  Commodity Ethernet is available
in a set of discrete bandwidth levels---1 Gbps, 10 Gbps, and 40 Gbps---with
cost increasing proportional to throughput (see
Table~\ref{tab:resourcesummary}).  Given our choice of 7.2k-RPM disks for
storage, a 1 Gbps network can accommodate at most one disk per server without
the network throttling disk I/O.  Therefore, we settle on a 10 Gbps network; 40
Gbps Ethernet has yet to mature and hence is still cost-prohibitive.  To
balance a 10 Gbps network with disk I/O, we use a server that can host 16
disks.  Based on the options available commercially for such a server, we use a
server that hosts 16 disks and 8 CPU cores.  The choice of 8 cores was driven
by the available processor packaging: two physical quad-core CPUs.  The larger
the number of separate threads, the more stages that can be isolated from each
other.  In our experience, the actual speed of each of these cores was a
secondary consideration, since the workloads we consider are mostly heavily
I/O-bound.

Third, our problem domains require both significant capacity and I/O
requirements from storage, since tens to hundreds of TB of data is to be stored
and all the data is to be read and written twice.  To determine the best
storage option given these requirements, we survey a range of hard disk options
shown in Table~\ref{tab:resourcesummary}.  We find that 7.2k-RPM SATA disks
provide the most cost-effective option in terms of balancing cost per GB and
cost per read/write MBps (assuming we can achieve streaming I/O).  To allow 16
disks to operate at full streaming I/O throughput, we require storage
controllers that are able to sustain at least 1600 MBps of streaming bandwidth.
Our hardware design necessitated two 8x PCI drive controllers, each supporting
8 disks, because of the PCI bus' bandwidth limitations.

The final design choice in provisioning our cluster is the amount of memory
each server should have.  The primary purpose of memory in our system is to
enable large amounts of data buffering so that we can read from and write to
the disk in large chunks.  The larger these chunks become, the more data can be
read or written before seeking is required.  We initially provisioned each of
our machines with 12 GB of memory; however, during development we realized that
24 GB was required to provide sufficiently large writes, and so the machines
were upgraded.  One of the key takeaways from our work is the important role
that buffering plays in enabling high utilization of the network, disk, and
CPU, and the efficient, dynamic management of that buffering is a key
contribution of this work.

The cluster we used for the research described in this dissertation consists of
70 HP DL380G6 servers, each with two Intel E5520 CPUs (2.27 GHz), 24 GB of
memory, and 16 500GB 7,200 RPM 2.5" SATA drives. Each hard drive is configured
with a single XFS partition. Each server has two HP P410 drive controllers with
512MB on-board cache, as well as a Myricom 10 Gbps network interface. The
network interconnect we used to evaluate TritonSort is a 52-port Cisco Nexus
5020 datacenter switch. During the development of Themis, we upgraded the
switch to a Cisco Nexus 5596UP.

\section{Software Architecture}

TritonSort and Themis are staged, pipeline-oriented dataflow processing
systems. Both systems are implemented as directed graphs of \emph{stages}. Each
stage implements part of the data processing pipeline and either sources,
sinks, or transmutes data flowing through it.

Each stage is implemented by a collection of \emph{workers}, each of which is a
separate thread. Workers receive input \emph{work units}, which are typically
in-memory buffers, by dequeuing them from a collection of per-stage queues. In
the process of running, a worker can produce work for a downstream stage, and
optionally direct the worker to which that work should be directed. If a worker
does not specify a destination worker, work units are assigned according to a
per-stage work queueing policy that defaults to round-robin. All workers in a
given stage graph run in parallel.

When a work unit arrives, the worker executes a stage-specific \texttt{run()}
method that implements the specific function of the stage. Workers process work
in one of three ways. First, a worker can accept an individual work unit,
execute the \texttt{run()} method over it, and then wait for new work. Second,
it can accept a batch of work (up to a configurable size) that has been
enqueued to one of its stage's queues. Lastly, it can keep its \texttt{run()}
method active, polling for the presence of new work units
explicitly. TritonSort and Themis contain examples of each of these three kinds
of methods.

To maximize cluster resource utilization, we need to design an appropriate
software architecture.  There are a range of possible software architectures in
keeping with our constraint of reading and writing every input tuple at most
twice.  The class of architectures upon which we focus share a similar basic
structure. These architectures consist of two phases separated by a distributed
barrier, so that all nodes must complete phase one before phase two begins.  In
the first phase, input data is read in parallel from the cluster's disks and
processed to produce intermediate data that is then routed to the node upon
which it will ultimately reside.  Each node is responsible for storing a
disjoint portion of the key space.  When data arrives at its destination node,
that node writes the data to its local disks.  In the second phase, each node
sorts the data on its local disks in parallel.  If running a MapReduce job, any
\reduce function processing occurs at this point. At the end of the second
phase, each node has a portion of the final output data set stored on its local
disks. In the case of sort, the sorted output partitions stored on all nodes
can be concatenated together to form the final sorted sequence.

\chapter{Related Work}
\label{chapter:related}

\section{Large-Scale Sorting Systems}

The Datamation sorting benchmark\cite{datamation} initially measured the
elapsed time to sort one million records from disk to disk. As hardware has
improved, the number of records required by the benchmark has grown to its
current level of 100TB.  Over the years, numerous authors have reported the
performance of their sorting systems, and we benefit from their
insights\cite{DEMSort, TokuSampleSort, SCS, nowsort, NSort, alphaSort}.  We
differ from previous sort benchmark holders in that we focus on maximizing both
aggregate throughput and per-node efficiency.

NOWSort\cite{nowsort} was the first of the aforementioned sorting systems to
run on a shared-nothing cluster.  NOWSort employs a two-phase pipeline that
generates multiple sorted runs in the first phase and merges them together in
the second phase, a technique shared by DEMSort\cite{DEMSort}.  An evaluation
of NOWSort done in 1998\cite{balance98} found that its performance was
limited by I/O bus bandwidth and poor instruction locality.  Modern PCI buses
and multi-core processors have largely eliminated these concerns; in practice,
\tritonsort is bottlenecked by disk bandwidth.


\section{Achieving Per-Resource Balance}

Achieving per-resource balance in a large-scale data processing system is the
subject of a large volume of previous research dating back at least as far as
1970.  Among the more well-known guidelines for building such systems are the
Amdahl/Case rules of thumb for building balanced systems~\cite{amdahlcase} and
Gray and Putzolu's ``five-minute rule''~\cite{fiveminuterule} for trading off
memory and I/O capacity.  These guidelines have been re-evaluated and refreshed
as hardware capabilities have increased.

\section{Architectural Influences}

The staged, pipelined dataflow architecture used in both TritonSort and Themis
is inspired in part by SEDA\cite{seda}, a staged, event-driven software
architecture that decouples worker stages by interposing queues between them.
Other DISC systems such as Dryad~\cite{dryad} export a similar model, although
Dryad has fault-tolerance and data redundancy capabilities that TritonSort and
Themis do not currently implement.

Many of our design decisions are informed by lessons learned from parallel
database systems.  Gamma\cite{gamma} was one of the first parallel database
systems to be deployed on a shared-nothing cluster.  To maximize throughput,
Gamma employs horizontal partitioning to allow separable queries to be
performed across many nodes in parallel, an approach that is similar in many
respects to our use of logical disks.  TritonSort's \sender-\receiver pair is
similar to the exchange operator first introduced by Volcano\cite{volcano} in
that it abstracts data partitioning, flow control, parallelism and data
distribution from the rest of the system.

\section{Fault Tolerance Techniques}

There is a large continuum of fault tolerance options between task-level and
job-level fault tolerance.  Percolator~\cite{percolator} provides
ACID-compliant transactions with snapshot-isolation semantics on its
multi-petabyte document repository. Checkpointing and rollback is another
popular form of fault tolerance; we refer the reader
to~\cite{Elnozahy:2002:SRP:568522.568525} for a survey of different techniques
in this space.  FLuX~\cite{flux} uses process-pairs replication to ensure that
if one of the two replicas fails, data processing can still continue seamlessly.

Several efforts have been made to increase the resilience of intermediate data
without dramatically impacting performance. ISS~\cite{ko-intermediate} provides
a replicated storage layer that increases the failure resilience of
intermediate and output data by asynchronously replicating it.  HOP~\cite{hop}
pipelines the transmission of intermediate data from map tasks to reduce tasks
with its materialization to local disk, only acting on optimistically
transmitted data when it has been ``committed'' at the source.

Lineage has long been of interest to a
wide range of fields, in areas as diverse as ensuring that research results can
be reproduced~\cite{Bose05lineageretrieval}, determining which source records
contributed to a record in a materialized view~\cite{cui_lineage}, and policy
enforcement~\cite{Xu06taint-enhancedpolicy}. Spark~\cite{spark} uses lineage at
the RDD level to provide fault tolerance for RDDs.

Recovery-Oriented Computing (ROC)~\cite{microreboot,roc} is a research vision
that focuses on efficient recovery from failure, rather than focusing
exclusively on failure avoidance.  This is helpful in environments where
failure is inevitable, such as data centers.  The design of task-level fault
tolerance in existing MapReduce implementations shares similar goals with the
ROC project.

\section{Multi-Query Optimization and Scan Sharing}

In the MapReduce context, multi-query optimization typically focuses on
reducing the number of I/O operations required to execute a set of jobs.
Agrawal, Kifer and Olson~\cite{ako08}'s scheduling approach for MapReduce
decides whether to try to delay jobs for possible scan sharing based on a model
of job arrival times and input file access patterns.
Circumflex~\cite{circumflex} builds upon this work by relaxing some of the
modeling assumptions.  In ~\cite{upenn-scanshare}, Zhang proposes a cost
function for estimating the savings from scan sharing.  MRShare~\cite{mrshare}
applies multi-query optimization to Hadoop, rewriting jobs that arrive in
batches so that they share input data scans.

\section{Improving MapReduce's Performance}

Several efforts aim to improve MapReduce's efficiency and performance.  Some
focus on runtime changes to better handle common patterns like job
iteration~\cite{haloop}, while others have extended the programming model to
handle incremental updates~\cite{CBP,percolator}.  Work on new MapReduce
scheduling disciplines~\cite{LATE} has improved cluster utilization at a map-
or reduce-task granularity by minimizing the time that a node waits for
work. Tenzing~\cite{tenzing}, a SQL implementation built atop the MapReduce
framework at Google, relaxes or removes the restriction that intermediate data
be sorted by key in certain situations to improve performance.

Massively parallel processing (MPP) databases often perform
aggregation in memory to eliminate unnecessary I/O if the output of that
aggregation does not need to be sorted.  Themis could skip an entire read and
write pass by pipelining intermediate data through the \reduce function
directly if the \reduce function was known to be commutative and
associative. We chose not to do so to keep Themis's operational model
equivalent to the model presented in the original MapReduce paper.

\section{Skew Mitigation in MapReduce}

Characterizing input data in both centralized and distributed contexts has been
studied extensively in the database systems
community~\cite{Manku99,DataSkeletons,Hadjieleftheriou2005}, but many of the
algorithms studied in this context assume that records have a fixed size and
are hence hard to adapt to variably-sized, skewed records. Themis's skew
mitigation techniques bear strong resemblance to techniques used in MPP
shared-nothing database systems~\cite{DeWittGraySkew}.

The original MapReduce paper~\cite{mapreduce} acknowledges the role that
imbalance can play on overall performance, which can be affected by data skew.
SkewReduce~\cite{SkewReduce} alleviates the computational skew problem by
allowing users to specify a customized cost function on input records.
Partitioning across nodes relies on this cost function to optimize the
distribution of data to tasks.  SkewTune~\cite{SkewTune} proposes a more
general framework to handle skew transparently, without requiring hints from
users.  SkewTune is activated when a slot becomes idle in the cluster, and
the task with the greatest estimated remaining time is repartitioned to take
advantage of that slot.  This reallocates the unprocessed input data through
range-partitioning, similar to Themis's phase zero.

Sailfish~\cite{sailfish} aims to mitigate partitioning skew in MapReduce by
choosing the number of reduce tasks and intermediate data partitioning
dynamically at runtime. It chooses these values using an index constructed on
intermediate data. Sailfish and Themis represent two design points in a space
with the similar goal of improving MapReduce's performance through more
efficient disk I/O.

\chapter{TritonSort: I/O-Efficient Large-Scale Sorting}
\label{chapter:tritonsort}

\chapter{Themis: I/O-Efficient MapReduce}
\label{chapter:themis}

\def\themis{Themis\xspace}

\chapter{I/O-Efficient Fault Tolerance}
\label{chapter:fault_tolerance}

A key requirement when building scale-out data processing architectures is
allowing them to recover from failures in a manner that is transparent to the
end user. Traditional MapReduce implementations provide fault tolerance by
materializing intermediate data to disk on both sides of a network
transfer. This increases the amount of disk I/O that each MapReduce job must
perform, which fundamentally limits the performance of I/O-bound workloads.

In this chapter, we argue that small and medium clusters -- on which MapReduce
is commonly deployed, and where the likelihood of a failure during a job is low
relative to large-scale clusters -- can benefit from more optimistic forms of
fault tolerance for which the common-case overhead is far lower than
traditional approaches. In particular, we explore the implications of the
job-level fault tolerance approach adopted by Themis in the previous chapter,
and describe an alternative fault tolerance method for Themis that leverages
prior work in scan sharing and eager record-level provenance.


\section{Introduction}

A key requirement and challenge in building scale-out data processing
architectures is allowing them to recover from failures without burdening the
programmer. MapReduce traditionally provides fault tolerance by splitting the
execution of the \map and \reduce functions into a collection of idempotent
\emph{tasks}. Each map task operates over a portion of the input, while each
reduce task operates over records produced by the \map function with a
particular set of keys. When a task fails, it is simply re-executed. We refer
to this method of fault tolerance as ~\emph{task-level fault tolerance}.

A key benefit of this fault tolerance technique is that it is
\emph{proportional}. Generally speaking, this means that the amount of
additional work required to recover from a failure is proportional to the size
of that failure. Proportional fault tolerance techniques work extremely well on
clusters containing thousands of nodes, because failures in those environments
are extremely common and the relative size of each individual failure is
small~\cite{jeff-dean-talk}.

In MapReduce's case, however, proportional fault tolerance comes with a
significant cost; map tasks must materialize their output to their local disks
before transferring that output to reduce tasks. These materializations are
required because, in general, each reduce task needs some of the records
produced by every map task in order to run. Were map tasks to send their
outputs to reduce tasks directly, the loss of the node on which a reduce
task runs would require that map tasks re-compute all data sent to that
task. In I/O-bound applications, the extra materializations required by
task-level fault tolerance can negatively
affect performance.

Many modern MapReduce clusters are ``dense'', in the sense that they pack a
large amount of storage, compute, and network bandwidth into a small number of
racks of servers. In this chapter, we show that in these ``dense'' clusters,
the additional I/O necessitated by task-level fault tolerance often leads to
lower overall job throughput than simply re-running a job if a failure
occurs.

The more optimistic \emph{job-level fault tolerance} employed by \themis in the
previous chapter allows \themis to perform much more aggressive operator
pipelining than task-level fault tolerance can achieve while still maintaining
the 2-IO property. However, job-level fault tolerance precludes running jobs
that take longer than the cluster MTTF to complete, preventing large clusters
(or unusually failure-prone small ones) from running some jobs.  To mitigate
this problem, we present a fault tolerance approach that provides proportional
recovery without imposing additional intermediate data materialization during
failure-free execution. Our main goal in designing this fault tolerance scheme
is to perform as little additional I/O as possible both in common case
operation and during recovery from failure.

Our contributions are as follows:

\begin{enumerate}
  \item We explore the tradeoffs of different levels of fault tolerance in
    ``dense'' clusters.
  \item We modify \themis to allow it to run multiple jobs concurrently, using
    scan sharing~\cite{nova, qptmd, coscan} to reduce the amount of I/O
    required for each job.
  \item Leveraging this multi-tenant capability, we present a fault tolerance
    mechanism that composes previously known techniques to reduce the amount of
    additional I/O needed for recovery at the expense of additional redundant
    computation.
  \item We show how this fault tolerance mechanism can be used to provide
    proportional recovery both from failures of a single disk and an entire
    node. When scan sharing, an eight-node \themis cluster can recover from a
    disk failure with under 5\% overhead.
  \item We compare this approach to approaches based on record-based provenance
    information.
\end{enumerate}

\section{Motivation}
\label{sec:motivation}
\label{sec:assumptions}

In this section, we summarize the argument for replacing task-level fault
tolerance as MapReduce's fault tolerance scheme for ``dense'' clusters. We then
provide an overview of alternative fault approaches used by current data
processing systems.

\subsection{Fault Tolerance for ``Dense'' Clusters}
\label{sec:ft_provenance}

\begin{figure*}[t]
\centering
\begin{subfigure}[t]{0.47\textwidth}
  \centering
  \includegraphics[width=\textwidth]{themis/graphs/analytical_failure_motivation/factor_5min}
  \caption{\label{fig:ft_motivation5} 5-minute job}
\end{subfigure}
\begin{subfigure}[t]{0.47\textwidth}
\centering
\includegraphics[width=\textwidth]{themis/graphs/analytical_failure_motivation/factor_60min}
\caption{\label{fig:ft_motivation60} 1-hour job (see text below for explanation of marked point)}
\end{subfigure}
\begin{subfigure}[t]{0.47\textwidth}
\centering
\includegraphics[width=\textwidth]{themis/graphs/analytical_failure_motivation/factor_600min}
\caption{\label{fig:ft_motivation600}10-hour job}
\end{subfigure}
\caption{\label{fig:ft_motivation} A lower-bound of the expected benefit of
  job-level fault tolerance for varying job durations and cluster sizes. Given
  that an error-free execution of a job with task-level fault tolerance takes
  five minutes \subref{fig:ft_motivation5}, an hour \subref{fig:ft_motivation60},
  or ten hours \subref{fig:ft_motivation600} to complete, we explore the
  expected performance improvement gained from job-level fault tolerance if an
  error-free run of the job executes 1, 2, 4, and 8x faster with job-level
  fault tolerance than it does with task-level fault tolerance.}
\end{figure*}

Much of MapReduce's architecture is based on the assumption that it is running
on a very large cluster of unreliable machines.  However, a large number of
``Big Data'' clusters do not approach the size of warehouse-scale data centers
like those at Google and Microsoft because moderately-sized clusters (10s of
racks or fewer) are increasingly able to support important real-world problem
sizes.  The storage capacity and number of CPU cores in commodity servers are
both increasing rapidly.  In Cloudera's reference system
design~\cite{ClouderaDellHadoopPlatform}, in which each node has 16 or more
disks, a petabyte worth of 1TB drives fits into just over three racks, or about
60 nodes.  Coupled with the emergence of affordable 10 Gbps Ethernet at the end
host and increasing bus speeds, data can be packed more densely than ever
before while keeping disk I/O as the bottleneck resource.  This implies that
fewer servers are required for processing large amounts of data for I/O-bound
workloads.  We now consider the implications of this increased density on fault
tolerance.

Job-level fault tolerance allows for much more aggressive operator pipelining
than task-level fault tolerance can achieve while still maintaining the 2-IO
property.  However, it is not self-evident that the overhead of re-executing
failed jobs does not cancel any performance gained by this aggressive
pipelining.  In this section, we show not only that job-level fault tolerance
is a feasible approach for moderately-sized clusters, but also that there are
significant potential performance benefits for using job-level fault tolerance
in these environments.

Understanding the expected impact of failures is critical to choosing the
appropriate fault tolerance model.  MapReduce was designed for clusters of many
thousands of machines running inexpensive, failure-prone commodity
hardware~\cite{mapreduce}.  For example, Table~\ref{tbl:googleMttf} shows
component-level mean-time to failure (MTTF) statistics in one of Google's data
centers~\cite{google-availability:osdi10}. Google's failure statistics are
corroborated by similar studies of hard
drive~\cite{DBLP:conf/fast/PinheiroWB07,Schroeder:2007:UDF:1288783.1288785} and
node~\cite{DBLP:conf/nsdi/NathYGS06, Schroeder:2010:LSF:1916484.1916652}
failure rates.

\subsection{Modeling Node Failure Rates}

At massive scale, there is a high probability that some portion of the cluster
will fail during the course of a job.  To understand this probability, we
employ a simple model~\cite{sysreliability}, shown in
Equation~\ref{eqn:jobfailure}, to compute the likelihood that a node in a
cluster of a particular size will experience a failure during a job:

\begin{equation}
P(N, t, MTTF) = 1 - e^{-N \cdot t / MTTF}
\label{eqn:jobfailure}
\end{equation}

The probability of a failure occurring in the next $t$ seconds is
a function of (1) the number of nodes in the cluster, $N$, (2) $t$, and (3) the
mean time to failure of each node, $MTTF$, taken from the node-level failure
rates in Table~\ref{tbl:googleMttf}.  This model assumes that nodes fail with
exponential (Pareto) probability, and we simplify our analysis by considering
node failures only.  We do this because disk failures can be made rare by using
node-level mechanisms (i.e., RAID), and correlated rack failures are likely to
cripple the performance of a cluster with only a few racks.

Based on the above model, in a 100,000 node data center, there is a 93\% chance
that a node will fail during any five-minute period. On the other hand, in a
moderately-sized cluster (e.g., 200 nodes, the average Hadoop cluster size as
reported by Cloudera), there is only a 0.53\% chance of encountering a node
failure during a five-minute window, assuming the MTTF rates in
Table~\ref{tbl:googleMttf}.

This leads to the question of whether smaller deployments benefit from
job-level fault tolerance, where if any node running a job fails the entire job
restarts.  Intuitively, this scheme will be more efficient overall when
failures are rare and/or jobs are short.

\subsection{Modeling Expected Job Completion Time}

Let $T$ be the job's duration and $MTTF$ be the mean time to failure of the
cluster. In our model, failure occurs as a Poisson process. We compute the
expected running time of a failed job (denoted $T_F$) as follows:

\vspace{-4mm}

\begin{eqnarray}
T_{F} &=& \int_{0}^{T} t \cdot \frac{1}{MTTF} e^{-\frac{t}{MTTF}} dt \notag \\
      &=& \left[- t e^{-\frac{t}{MTTF}} - MTTF \cdot e^{-\frac{t}{MTTF}}\right]_{t = 0}^{t = T} \notag \\
      &=& MTTF - (T + MTTF) e^{-\frac{T}{MTTF}}
\label{eqn:T_F}
\end{eqnarray}

Therefore, if the job duration $T$ is much larger than the MTTF of the cluster
($T \gg MTTF$), Equation~\ref{eqn:T_F} implies that $T_F \approx MTTF$, and we
expect the job to fail. On the other hand, if $T \ll MTTF$,
Equation~\ref{eqn:T_F} implies that $T_F \approx T$, and we expect the job to
succeed.

Having modeled the running time of a failed job, we can now derive a model for
overall job completion time. Let $p$ denote the probability of failure in a
single Themis job.  Let $T$ denote the running time of the job when there are
no failures.

Consider a situation in which the job fails during the first $(n-1)$ trials and
completes in the $n^{th}$ trial. The probability of this event is $p^{n-1} (1 -
p)$.  Note that a successful trial takes time $T$ and a failed trial takes time
$T_F$.  To simplify our notation, let $\alpha = T_F / T$ be the fraction of its
successful runtime the failed job spent running.  Then the total running time
in this case is
\[(n-1) \alpha T + T = ((n-1) \alpha + 1) T.\]

By considering such an event for all possible values of $n$, we get the
expected running time to completion for the job:

\vspace{-4mm}

\begin{eqnarray}
   S(p, T) &=& \sum_{n=1}^{\infty} ((n-1) \alpha + 1) T \cdot p^{n-1}(1-p)  \notag \\
           &=& T(1-p) \sum_{n=1}^{\infty} \left(\alpha n p^{n-1} + (1-\alpha) p^{n-1}\right)  \notag \\
           &=& T(1-p) \left(\alpha \sum_{n=1}^{\infty} n p^{n-1} + (1 - \alpha) \sum_{n=1}^{\infty} p^{n-1}\right) \notag \\
           &=& T(1-p) \left(\alpha \frac{1}{(1-p)^2} + (1 - \alpha) \frac{1}{1-p}\right) \notag \\
           &=& T \left(\alpha \frac{p}{1-p} + 1\right)
\label{eqn:job}
\end{eqnarray}

Hence, we can model the expected completion time of a job $S(p,T)$ as:

\begin{equation}
S(p,T) = T\left(\frac{p}{1 - p} + 1\right)
\label{eqn:expectedjobtime}
\end{equation}

\begin{table}[t]
\centering
\caption{\label{tbl:googleMttf} Component-level failure rates
observed in a Google data center as reported
in~\cite{google-availability:osdi10}.}
\begin{tabular}{|c|c|} \hline
\textbf{Component} & \textbf{Failure rates}\\\hline
Node & 4.3 months \\
Disk & 2-4\% annualized\\
Rack & 10.2 years \\\hline
\end{tabular}
\end{table}

where $p$ is the probability of a node in the cluster failing, and $T$ is the
runtime of the job.  This estimate is pessimistic, in that it assumes that jobs
fail just before the end of their execution.

By combining equations~\ref{eqn:jobfailure} and \ref{eqn:expectedjobtime}, we
can compute the expected benefit--or penalty--that we get by moving to
job-level fault tolerance.  Modeling the expected runtime of a job with
task-level fault tolerance is non-trivial, so we instead compare to an
error-free baseline in which the system's performance is not affected by node
failure.  This comparison underestimates the benefit of job-level fault
tolerance.

Figure~\ref{fig:ft_motivation} shows the expected performance benefits of
job-level fault tolerance compared to the error-free baseline.  More formally,
we measure performance benefit as $S(P(\cdot),T_{job}) / T_{task}$,
where $T_{job}$ is the time a job on an error-free cluster takes to execute
with job-level fault tolerance and $T_{task}$ is the time the same job takes to
execute with task-level fault tolerance.

The benefits of job-level fault tolerance increase as the error-free
performance improvement made possible by moving to job-level fault tolerance
(i.e. $T_{task} / T_{job}$) increases. For example, if $T_{task} / T_{job}$ is
4, $T_{task}$ is one hour and we run on a cluster of 1,000 nodes, we can expect
\themis to complete the job 240\% faster than the task-level fault tolerant
alternative on average; this scenario is marked with a star in
Figure~\ref{fig:ft_motivation60}.  There are also situations in which
job-level fault tolerance will significantly under-perform task-level fault
tolerance.  For example, if $T_{task} / T_{job}$ is 2, \themis will
under-perform a system with task-level fault tolerance for clusters bigger than
500 nodes.  From this, we make two key observations: for job-level fault
tolerance to be advantageous, the cluster has to be moderately-sized, and
\themis must significantly outperform the task-level alternative.

\section{Alternative Fault Tolerance Methods}
\label{sec:fault_tolerance_approaches}

We now examine a number of alternative fault tolerance schemes and
their applicability to ``dense'' clusters.

\subsection{Replication}

Systems that employ replication for fault tolerance store multiple copies, or
replicas, of intermediate data in the system simultaneously. The granularity of
this replication can vary: whole files, the blocks that comprise a file, or
even individual records may be replicated. To prevent correlated failures from
causing data loss, these replicas are often stored in different failure
domains; for example, replicas might be stored on different hosts, different
racks, or even geographically-separated data centers. If one of the replicas is
lost, another replica can be used in its place, either by migrating it or
accessing it remotely.

While MapReduce typically relies on some degree of replication in its input and
output data for fault tolerance, intermediate data generated by individual \map
tasks is typically not replicated due to the high overhead both in terms of
storage space and bandwidth involved (though Ko et al. explore mitigating these
effects in ~\cite{ko-intermediate}).

\subsection{Upstream backup}

The task-level fault tolerance scheme currently used in MapReduce is an example
of a class of fault tolerance called \textit{upstream
  backup}~\cite{magda-ft,spark}.  In upstream backup, the output of an operator
is buffered locally on disk before being sent over the network to subsequent
operators.  If the downstream operator fails (due to node failure, for
example), its inputs can be sent to a replacement instance of the operator
without having to re-run the map tasks that generated those inputs.  Upstream
backup is a restricted form of keeping a bounded history in dataflow
systems~\cite{magda-ft}.

\subsection{Parallel Recovery}

A disadvantage of upstream backup is that the
recovery latency can be high because recovery of a downstream operator is
limited to the speed at which the slowest upstream node can send data to it.
In parallel recovery, intermediate data is additionally checkpointed on many
separate nodes.  When a failure occurs, each of these nodes can contribute a
small portion of the recovery data to the new downstream operator.  This
enables significant parallelism, reducing the time required to recover the
data. Spark's D-Streams~\cite{dstreams} and RAMCloud~\cite{ramcloud-ft} both
employ parallel recovery.

\subsection{Process-Pairs}

In systems employing process-pairs
parallelism~\cite{gray-reuter}, two replicas of the same computation are
executed simultaneously. In the traditional definition, checkpoints of the
primary's execution are periodically sent to the backup and, if the primary
fails, the backup assumes the primary's role and a new backup is
instantiated. FLuX~\cite{flux} applies the process-pairs approach to the
continuous query domain, providing process pairs on either side of a network
transmission and providing seamless fail-over. While this approach allows the
computation to continue in the face of a limited amount of failure, it
potentially imposes significant additional network bandwidth and compute
overhead.

\subsection{Provenance and Selective Replay}

The above mechanisms work to
ensure that data itself is kept fault-tolerant. Fault tolerance mechanisms
based on provenance and replay ensure instead that the sequence of steps
necessary to reproduce each piece of intermediate data are kept fault tolerant,
while the intermediate data itself is volatile. MapReduce employs a limited
form of provenance; a map task's output can be recomputed if the function that
task was running and the data over which it was running are known, without
recomputing anything else.

The storage requirements of maintaining provenance information depend largely
on its granularity. For example, the overhead of record-level provenance is a
function of the number of intermediate records, which can be quite large at
scale.  However, provenance can be quite effective when kept at a much
coarser-grained level.  Spark maintains provenance at the Resilient Distributed
Dataset (RDD) level, which requires much less overhead than record-level
bookkeeping. However, the Spark authors point out that they perform upstream
backup of intermediate records for what they call ``wide dependencies'', of
which MapReduce's all-to-all shuffle is one, ``... to simplify fault
recovery''~\cite{spark}.

\subsection{Scan-Sharing}

Scan-sharing~\cite{qptmd} is a form of multi-query optimization
in which the output of a scan of a dataset is used by more than one job at a
time. This optimization takes advantage of the fact that some datasets are much
more popular than others. Jobs that share the same data can be co-scheduled and
``share'' scans of that data, effectively eliminating the I/O overhead for all
but one of the jobs.  For I/O-bound workloads, this provides a significant
reduction in overhead, and does not require any additional storage overhead or
maintenance of provenance.

\section{Design}
\label{fault_tolerance:sec:design}

In this section, we describe our goals in implementing fault tolerance for
``dense'' MapReduce clusters. We then present an overview of the design of our
fault tolerance approach, which incorporates aspects of several of the
approaches described in Section~\ref{sec:fault_tolerance_approaches}.

\subsection{Goals}

Our goals when designing a fault tolerance scheme for ``dense'' MapReduce
clusters are as follows. First, recovery should be proportional; that is, the
amount of additional time taken to recover from a failure should be
proportional to the failure's size. Second, the fault tolerance scheme should
impose as little additional disk I/O in failure-free operation as possible, and
perform as little additional disk I/O during recovery as possible. Finally, the
system should be able to recover from failures of both a disk and an entire
node.

\subsection{Recovery in MapReduce}

In this work, we assume that failures are fail-stop with complete loss of
state. This means that if a disk fails, all data stored on that disk is lost.
If a node fails, all its disks are considered to have failed. Failed disks and
nodes must be explicitly recovered by an operator. Recovering from Byzantine
faults is beyond the scope of this work.

Fundamentally, recovering from a failure in MapReduce consists of two main
tasks. Any intermediate data that was stored on failed disks must be
recovered. We call this part of the recovery process \emph{write recovery},
because it ensures that all intermediate records have been written. Also, the
system must ensure that all input data was completely processed. If a node was
in the middle of processing an input file when it failed, some of that file's
records may not have been mapped and transmitted successfully. We call this
part of the recovery process \emph{read recovery}, because it ensures that
every input record has been read and mapped.

\subsection{Write Recovery Approach}

In order to perform write recovery, the system must regenerate all intermediate
data that was supposed to have been stored on the failed disks. \themis uses a
technique we call \emph{scan-and-discard} to perform this recovery. In the
scan-and-discard approach, the input data set is re-read and each record is
re-mapped, but only those records that would have been stored on the failed
disks are transmitted to their destination.

One obvious drawback of the scan-and-discard technique is that all input
records must be re-read and re-mapped, even though most of those records will
not be transmitted. \themis attempts to reduce or eliminate this additional I/O
cost through scan sharing.

There is a large body of prior work suggesting both a significant opportunity
for and potential benefit from scan sharing in the MapReduce context. Recent
traces from industrial MapReduce deployments~\cite{Chen2012} indicate that
there are many opportunities for scan sharing in multi-tenant MapReduce
clusters. In these traces, input file access frequency is roughly Zipfian,
meaning that most input file accesses are for a small number of ``hot''
files. In addition, input file access exhibits a large amount of temporal
locality. In the traces analyzed in \cite{Chen2012}, between 60 and 90\% of
input file re-accesses happen within one hour of the original access. In one
particular workload (a Cloudera customer running a cluster of 100 machines),
70\% of input re-accesses occurred within one minute of the original access.
Agrawal, Kifer and Olson~\cite{ako08} observe that there are often many
concurrent jobs that access a shared set of data files. The authors of
Comet~\cite{comet} achieved a 50\% reduction in total I/O in their DryadLINQ
cluster using scan sharing. Scan sharing has also been shown to provide a
significant improvement in job throughput for Pig and Hive
workloads~\cite{nova, coscan, query-opt-mapreduce}.

\subsection{Read Recovery Approach}

Our approach to read recovery is similar to that for write recovery; we re-read
any input files that may not have been completely processed and re-map each
record. In contrast to our write recovery approach, only records that the
failed node would have sent to the remaining live nodes are transmitted to
their destinations.

Once the read recovery process has completed, each intermediate record is
guaranteed to be present on the cluster's intermediate disks at least once.  To
maintain correctness, however, the \reduce function must not reduce multiple
duplicate copies of the same record, since this would likely change the result
of the job. Maintaining exactly one copy of each intermediate record is
challenging and potentially quite heavyweight, since it involves tracking
whether each intermediate record was successfully transmitted by the failed
node prior to the failure. We avoid this complication by allowing duplicates
and filtering them out on demand in a manner that is transparent to the \reduce
function.

\section{Implementation}

In this section, we describe the implementation of our fault tolerance
strategies as an extension to \themis. Section~\ref{sec:themis} provides a
brief overview of \themis' architecture, and Section~\ref{sec:recovery}
describes the implementation of our write and read recovery strategies in the
context of that architecture. In Section~\ref{sec:multi-tenancy}, we describe
extensions to \themis to support multi-tenancy. Section~\ref{sec:control_plane}
describes the way that jobs are dispatched.
Section~\ref{sec:input_file_gathering} describes how files are assigned to
nodes, and explores the practical concern of achieving high bandwidth from
distributed storage. Section~\ref{sec:fault_response} describes how failures
are detected and how nodes respond to failure during a job.

\subsection{Themis: I/O-Efficient MapReduce}
\label{sec:themis}

In this section, we present a brief recap of the design of \themis, our highly
I/O-efficient MapReduce system. A more detailed description and evaluation of
\themis is presented in Chapter~\ref{chapter:themis}.  We opted to implement
our fault tolerance scheme in \themis rather than Hadoop because \themis lacked
a proportional fault tolerance mechanism prior to this work, whereas the
task-level fault tolerance scheme used by Hadoop is a tightly-integrated part
of its design.

\begin{figure}
  \centering
  \begin{subfigure}[t]{\columnwidth}
  \centering
  \includegraphics[width=\columnwidth]{fault_tolerance/figures/detailed_phase_one.pdf}
  \caption{\label{fig:phase_one} Phase One}
  \end{subfigure}\vspace{1em}
  \begin{subfigure}[t]{\columnwidth}
  \centering
  \includegraphics[width=\columnwidth]{fault_tolerance/figures/phase_two.pdf}
  \caption{\label{fig:phase_two} Phase Two}
  \end{subfigure}

  \caption{\label{fig:themis_phases} A diagrammatic overview of \themis' phases.}
\end{figure}

Nodes in a \themis cluster each have a collection of \emph{intermediate disks}
that store volatile intermediate data and a disjoint collection of \emph{DFS
  disks} that store input and output data, and are typically under the control
of a distributed file system like HDFS.

\themis runs a MapReduce job in two main \emph{phases}, called \emph{phase one}
and \emph{phase two}.  In phase one, input records are read in parallel from
the cluster's DFS disks. \themis applies the \map function to each record,
producing a collection of \emph{intermediate records} that are written to
intermediate partitions spread across the cluster's intermediate disks. Each
intermediate partition holds all records with a certain set of keys. The
mapping from keys to intermediate partitions is determined by a \emph{partition
  function}. Phase one is roughly analogous to Hadoop's map and shuffle phases.

At the end of phase one, all intermediate records have been generated,
partitioned and stored across the cluster's intermediate disks. A diagrammatic
overview of phase one is given in Figure~\ref{fig:phase_one}.

In phase two, each intermediate partition is read from the cluster's
intermediate disks completely into memory. Once in memory, it is sorted in-core
by key, and the \reduce function is applied to each group of records in the
partition with the same key. This produces a collection of \emph{output
  records} that are written to files on the DFS disks. Phase two is roughly
equivalent to Hadoop's sort and reduce phases. A diagrammatic overview of phase
two is given in Figure~\ref{fig:phase_two}.

Note that phase one requires all-to-all communication among cluster nodes, but
that phase two can be executed on each node independently.

\subsubsection{Partitioning}

In order for phase two to be processed efficiently, partitions should be small
enough for several of them to be processed in memory
simultaneously. Additionally, they should be as uniformly-sized as possible to
prevent stragglers. The partition function is responsible for ensuring both
these properties. The user can provide their own partition function, or it can
be derived at runtime through an optional sampling phase called \emph{phase
  zero}. Phase zero requires a fairly small sample to produce a good partition
function, and typically takes under a minute to run.


\subsection{Recovery Mechanism}
\label{sec:recovery}

As described in Section~\ref{fault_tolerance:sec:design}, recovering from a
failure consists of two central actions: write recovery and read recovery. In
the case of \themis, write recovery involves recovering partitions on any
intermediate disk that failed, while read recovery involves re-generating
missing pieces of partitions that a failed node should have produced, but
didn't. When a job fails, \themis will recover it by running a \emph{recovery
  job}, which is treated like a normal MapReduce job but is dedicated to
recovery.

\begin{figure}
  \centering
  \begin{subfigure}[t]{\columnwidth}
    \centering
    \includegraphics[width=\textwidth]{fault_tolerance/figures/disk_failure_before_recovery}
    \caption{\label{fig:disk_fail_before} The state of the job's intermediate
      partitions after the failure of disk 4. All intermediate partitions
      stored on disk 4 has been lost.}
  \end{subfigure}\hspace{0.05\textwidth}
  \begin{subfigure}[t]{\columnwidth}
    \centering
    \includegraphics[width=\textwidth]{fault_tolerance/figures/disk_failure_after_recovery}
    \caption{\label{fig:disk_fail_after} The state of the job's intermediate
      partitions after recovery from the disk failure. Intermediate data for
      the partitions on disk 4 have spread across disks 1 through 3.}
  \end{subfigure}
  \caption{\label{fig:disk_fail} Illustrative example of disk failure and
    recovery in a two-node cluster with two intermediate disks per node and
    eight intermediate partitions. The rectangles
    representing each partition are labeled with the disk or disks on which
    data for that partition is stored.}
\end{figure}

Before we explore the technical details of the implementation, consider the
following illustrative example. Suppose that \themis is running on a two-node
cluster with two intermediate disks each, storing a total of eight intermediate
partitions for a particular job. In Figure~\ref{fig:disk_fail_before}, disk 4
in this cluster has failed during phase one, causing the loss of partitions 7
and 8. Figure~\ref{fig:disk_fail_after} shows the state of the intermediate
partitions at the end of phase one of the recovery job, when the data for
partitions 7 and 8 has been recovered.

Note that the recovered data for partitions 7 and 8 is spread across all the
remaining disks roughly evenly; this is highly desirable because it allows as
many disks as possible to participate in phase two of the recovery job. It
should also be noted that after the failure of disk 4 in phase one, phase two
can be run to completion on partitions 1 through 6 without waiting for the
recovery job to recover the other partitions.

\begin{figure}[t]
  \centering
  \begin{subfigure}[t]{\columnwidth}
    \centering
    \includegraphics[width=\textwidth]{fault_tolerance/figures/node_failure_before_recovery}
    \caption{\label{fig:node_fail_before} The state of the job's intermediate
      partitions after the failure of node 2. All intermediate data for disks 3
      and 4 has been lost, and some data for the remaining partitions may not
      have been generated.}
  \end{subfigure}\hspace{0.05\textwidth}
  \begin{subfigure}[t]{\columnwidth}
    \centering
    \includegraphics[width=\textwidth]{fault_tolerance/figures/node_failure_after_recovery}
    \caption{\label{fig:node_fail_after} The state of the job's intermediate
      partitions after recovery from the node failure. Intermediate data for
      the partitions on disks 3 and 4 have spread across disks 1 and 2, and
      data that should have been produced by node 2 has been added to node 1's
      partitions (although there may be duplicates).}
  \end{subfigure}
  \caption{\label{fig:node_fail} Illustrative example of node failure and
    recovery in a two-node cluster with two intermediate disks per node and
    eight intermediate partitions. A `?' indicates that it is unknown
    whether the data has been lost or not.}
\end{figure}

Figure~\ref{fig:node_fail_before} shows the same cluster after experiencing a
failure of an entire node. Not only have partitions 5 through 8 been lost, but
the remaining partitions are incomplete because the node did not finish
producing intermediate data for those partitions before it failed. Once phase
one of the recovery job has completed in Figure~\ref{fig:node_fail_after},
write recovery has spread the lost data from partitions 5 through 8 across
disks 1 and 2, while read recovery has ensured that every record that belongs
in partitions 1 through 4 has been written at least once.

\begin{table*}[t]
  \centering
  \caption{\label{table:failure_response} Table summarizing \themis' response
    to various kinds of failures at different points in the job.}
  \resizebox{\columnwidth}{!}{
  \begin{tabular}{|c|c|c|c|c|}
    \hline
    \textbf{Phase} & \textbf{Failure} & \textbf{Write Recovery} & \textbf{Read Recovery} & \textbf{Run Subsequent Phases?} \\
    \hline
    Zero (Sample) & Any & None & None & Yes \\
    One (Map + Shuffle) & Disk & Failed disk's partitions & None & Yes \\
    One (Map + Shuffle) & Node & All node's disks' partitions & Node's input files & No \\
    Two (Sort + Reduce) & Disk & Failed disk's partitions & None & Yes \\
    Two (Sort + Reduce) & Node & All node's disks' partitions & None & Yes \\
    \hline
  \end{tabular}
}
\end{table*}

In contrast to disk failure, phase two cannot be run after a node failure in
phase one because some intermediate partitions may not be complete.

If a disk or node fails in phase two, write recovery must be performed to
restore the intermediate data that was lost in the failure, but no read
recovery must be performed. Since phase zero is optional and does not produce
any output aside from a partition function, any failure during phase zero
simply requires re-executing it.

The responses to various kinds of failure in each of \themis' stages is
summarized in Table~\ref{table:failure_response}.

In the following sections, we will describe the mechanisms behind both write
and read recovery.

\subsubsection{Write Recovery}

To perform write recovery, the recovery job must re-map the failed job's input,
discarding any records that were not stored on the intermediate disks that
failed. To do this, the recovery job wraps its partition function in a
\emph{record filter}. This filter is applied to each record before it is passed
to the partition function. Abstractly, a record filter is a function that takes
a record as input and returns either ``accept'' or ``reject''. The filter
accepts a record if the record belongs to one of the partitions being
recovered, and rejects it otherwise.

In practice, \themis accomplishes record filtering in one of two ways. If the
failed job was using a user-defined partition function, the record filter
applies the failed job's partition function to the record. If the resulting
partition number is outside the range of partitions to be recovered, the filter
rejects the record.

If the original job is using a partition function generated by phase zero, the
record filter stores a set of boundary key ranges, one per contiguous range of
partitions being recovered. The complete list of boundary keys for each
partition is stored on distributed storage at the end of phase zero, and the
filter retrieves the appropriate boundary keys when it is constructed.

When an intermediate record is emitted by the \map function, the filter first
compares each intermediate record's key to the boundaries of each of its
ranges; if the record is within any of the filter's ranges, the filter
accepts the record.

In order to speed recovery by writing to as many disks in parallel as possible,
the intermediate data being recovered is spread across the cluster's remaining
intermediate disks. This is done by running phase zero during the recovery job
on a filtered sample of the input data, which generates a partition function
that spreads data in the filtered partition ranges evenly throughout the cluster.

At the end of phase one of the recovery job, any partitions that were
completely lost during a failure have been reconstituted and spread across the
cluster's remaining intermediate disks.

\subsubsection{Read Recovery}

As Table~\ref{table:failure_response} illustrates, read recovery is always
performed alongside write recovery. We take advantage of this by piggy-backing
read recovery on write recovery.

In order to perform read recovery, we must first know the set of files that
were not completely processed by the failed node. \themis tracks which files
were completely mapped and received using a form of end-to-end
acknowledgments~\cite{endtoendargument}.  When a node is done reading a file in
phase one, it sends an EOF, or ``end-of-file'', annotation to every node in the
cluster indicating that the node will not receive any more data for the
file. Special care is taken to ensure that every intermediate record associated
with the file is transferred before this annotation. When a node receives an
EOF annotation, it adds the file's file ID to a list. At the end of phase one,
these lists are merged together to form a list of the nodes that received each
file. If every live node received an EOF annotation for a file, performing read
recovery on that file is not necessary. Each input file is checked for this
condition when constructing the input file list for a recovery job and files
are flagged for read recovery as appropriate.

Once phase one has completed, two sets of intermediate files will exist for the
failed job: the partially-complete set of files from the failed job and the set
of files generated as a result of read recovery. It is likely that some of the
records in these files are duplicates, and any duplicate records must be
removed to retain the \reduce function's correctness. To distinguish
intermediate records from one another, \themis tags each intermediate record
with \emph{source metadata} that uniquely identifies the record.

To uniquely identify intermediate records, we leverage the common assumption
that the \map function is deterministic and, as such, that applying the \map
function to an input record creates a totally-ordered sequence of intermediate
records. We identify an intermediate record by the position of its ``parent''
input record within the input dataset and its position in the totally-ordered
sequence of intermediate records. Specifically, we tag each record with a
64-bit file GUID, a 64-bit offset, and a 32-bit intermediate record ID. For the
purposes of evaluation, we store all 20 bytes of metadata even if the metadata
could potentially be compressed; note that for records with small offsets and
intermediate record IDs, these three pieces of metadata require far less than
20 bytes per record to store.

In phase two, intermediate partitions from the failed and recovery jobs with
the same intermediate partition number are concatenated together into an
in-memory buffer and sorted as a single intermediate partition. Before the
\reduce function is called on a set of intermediate records with a given key,
that set of records is secondarily sorted by its source metadata. The \reduce
function's record iterator then skips any records whose source metadata is the
same as that of the previous record.


\subsection{Multi-Tenancy in \themis}
\label{sec:multi-tenancy}

Each \themis node runs as a single process that assumes that it has exclusive
access to its intermediate disks and that it will not experience
memory pressure from other processes that results in swapping as long as it
does not exceed its configured memory limit. Its memory and disk management
subsystems (described in detail in~\cite{themis} and~\cite{tritonsort}) rely on
these assumptions and are the key enablers of \themis' I/O-efficiency and high
performance. Hence, running multiple \themis processes on a single node would
result in degraded performance since the processes would interfere with one
another.

\begin{figure}
  \centering
  \includegraphics[width=\columnwidth]{fault_tolerance/figures/multi_tenancy}
  \caption{\label{fig:multi_tenancy} An overview of multi-tenancy in
\themis. Input records are mapped by both job A and job B's \map functions,
and intermediate records are routed based on each job's partition function independently.}
\end{figure}

To allow multiple jobs to run simultaneously in \themis with minimal
interference, we have modified \themis to support running multiple jobs
concurrently within a single process. To allow for this concurrent processing,
records read from disk are passed through each job's \map function one function
at a time, but intermediate records are transferred and written in
parallel. Buffers of intermediate records produced by a \map function are
tagged with the unique ID of that \map function's job before being sent to the
appropriate destination node. Once a buffer is received, this job ID is used to
determine to which set of intermediate partitions the buffer's records will be
written. This process is illustrated in Figure~\ref{fig:multi_tenancy}

A unique feature of our deployment prototype is that it does not co-schedule
\map and \reduce function computation. Instead, it organizes jobs into
\emph{batches}, and runs phases one and two for all jobs in a batch
simultaneously before processing the next batch. If phase zero needs to be run
to compute partition functions for any of these jobs, it is run on each job in
the batch individually before phase one starts.


\subsection{Job Dispatch}
\label{sec:control_plane}

The execution of batches of jobs is controlled by a \emph{cluster
  coordinator}. The cluster coordinator accepts descriptions of batches from
clients and coordinates their execution across the cluster's machines. Each
machine in the cluster runs a \emph{node coordinator} that is responsible for
running a \themis process on its machine and reporting an error if it crashes.

Messages are exchanged between the user, the cluster coordinator and the node
coordinators through the manipulation of message queues. Additionally, the
coordinators maintain metadata about both themselves and the jobs they run. In
our current implementation, the role of message queues and metadata store are
both filled by a Redis~\cite{redis} database. Redis was chosen primarily for
convenience; a scalable key-value store like Hyperdex~\cite{hyperdex} or
Cassandra~\cite{cassandra} and message queue like Kafka~\cite{kafka} or
Kestrel~\cite{kestrel} could be substituted.

To run a batch, the user pushes a description of the jobs in the batch to the
cluster coordinator's job queue. Upon dequeueing a batch, the cluster
coordinator assigns a unique job ID to each job in the batch. It then determines
the set of input files that each job will process, and divvies those files out
among nodes. We describe this process in more detail in
Section~\ref{sec:input_file_gathering}.


\subsection{Input Files and Distributed Storage}
\label{sec:input_file_gathering}

\begin{figure}
  \centering
  \includegraphics[width=\columnwidth]{fault_tolerance/graphs/hdfs_no_proxy_penalty}
  \caption{\label{fig:hdfs_no_proxy_penalty} Comparing the performance of
    unmodified HDFS, HDFS with whole file replication for the primary replica,
    and reading and writing from raw disks.}
\end{figure}

The specification of each \themis job includes an input directory; all files in
the input directory are processed. \themis can read input files from raw disks
or from HDFS~\cite{hdfs} using the WebHDFS REST API. We use HDFS exclusively in
this work.

Each file is uniquely identified by its URL, which is of the form\\
\texttt{<protocol>://<host>:<port>/<path from root>}. Each file is also given a
file ID that must be unique within the job. In our implementation, a
file's ID is the upper 64 bits of the MD5 hash of its URL.

Our main concern when moving from raw disks to distributed storage was
maximizing the amount of bandwidth we could achieve from the storage system.
In order to achieve sufficient bandwidth, we found that we needed to change the
way HDFS allocates blocks for files. In particular, we modified HDFS so that it
performs \emph{whole-file replication} of the file's primary replica by placing
every block on a specific disk in the cluster based on the file's name. For
example, a file named \texttt{/1.2.3.4/3/<path>} would be stored on the third
DFS disk on node \texttt{1.2.3.4}. To allow \themis to remain oblivious to this
scheme, we implemented a proxy that performs a basic round-robin allocation of
primary file replicas to cluster disks and transparently maps between regular
and placement-aware paths. The proxy only interposes itself in communication
between \themis and HDFS when a file is first opened, and imposes no additional
overhead thereafter.

Figure~\ref{fig:hdfs_no_proxy_penalty} compares the performance of an 800GB, 8
node sort with and without these modifications; as a reminder, phase one of
\themis reads sequentially from HDFS, while phase two writes sequentially to
it. The substantial performance improvement for reads is the result of the
elimination of read contention on each node's DFS disks when many files are
being read simultaneously. However, the increased rigidity of block allocation
imposed by the proxy makes the performance of writes slightly worse than
unmodified HDFS.

We found that HDFS' block placement APIs were not sufficient for providing
whole-file replication for all of a file's replicas. Hence, blocks for all
other replicas are allocated according to HDFS' default policy, and access to
non-primary replicas occurs at the speed of unmodified HDFS. The cluster
coordinator will assign files to nodes that contain their primary replica
whenever possible.


\subsection{Responding to Failures}
\label{sec:fault_response}

As node coordinators run, they refresh a keep-alive key in Redis every few
seconds; if a node fails to refresh its keep-alive key, the cluster coordinator
presumes that the node has failed. A node notifies the cluster coordinator
directly if it finds that it can no longer write to one of its intermediate
disks.

\themis attempts to insulate the rest of the cluster from a failure whenever
one occurs so that the healthy portion of the cluster can complete as much work
as possible. To avoid the attendant complexity and fragility of coordinating
failure notification across nodes, \themis simply discards any data meant for a
failed portion of the cluster. When a node fails, all existing TCP sockets to
that node will break. Nodes respond to broken sockets by discarding all data
meant for that socket for the remainder of the batch. Similarly, when a disk
fails, all data that would have been written to the failed disk for the rest of
the batch is discarded. Subsequent batches will not use failed disks or nodes
until an operator has explicitly marked them as having recovered.

Currently, the user is responsible for issuing a recovery job to recover a
failed job. Scheduling recovery jobs to maximize the likelihood of scan sharing
is beyond the scope of this work; we examine some related efforts relevant to
this problem in Chapter~\ref{chapter:related_work}.


\section{Per-Record Replay Proportionality}
\label{sec:recovery_cost}

\begin{figure}
  \centering
  \includegraphics[width=\columnwidth]{fault_tolerance/graphs/scan_vs_seek}
  \caption{\label{fig:seek-vs-scan} Time to sequentially scan a 13.5 GB file
    vs. selectively reading a percentage of records.}
\end{figure}

\begin{figure*}
  \centering
  \begin{subfigure}[b]{\textwidth}
    \centering
    \includegraphics[width=\textwidth]{fault_tolerance/graphs/simultaneous_scan_same_file}
    \caption{\label{fig:simultaneous_same_file_scan} The effect of
      simultaneity on sequential scans.}
  \end{subfigure}
  \begin{subfigure}[b]{\textwidth}
    \centering
    \includegraphics[width=\textwidth]{fault_tolerance/graphs/simultaneous_seek_same_file}
    \caption{\label{fig:simultaneous_same_file_seek} The effect of
      simultaneity on selective reads.}
  \end{subfigure}
  \caption{\label{fig:simultaneous_same_file} The negative impact of
    both scanning through and selectively reading from the same file simultaneously.}
\end{figure*}

\begin{figure*}
  \centering
  \begin{subfigure}[b]{\textwidth}
    \centering
    \includegraphics[width=\textwidth]{fault_tolerance/graphs/simultaneous_scan_different_files}
    \caption{\label{fig:simultaneous_different_files_scan} The effect of
      simultaneity on sequential scans.}
  \end{subfigure}
  \begin{subfigure}[b]{\textwidth}
    \centering
    \includegraphics[width=\textwidth]{fault_tolerance/graphs/simultaneous_seek_different_files}
    \caption{\label{fig:simultaneous_different_files_seek} The effect of
      simultaneity on selective reads.}
  \end{subfigure}
  \caption{\label{fig:simultaneous_different_files} The negative impact of
    running a scan on one file while selectively reading records from a
    second file.}
\end{figure*}

When examining our options for adding fault tolerance to \themis, we were
particularly motivated by the idea of being able to recover by only reading and
re-mapping records whose intermediate data contributed to failed intermediate
partitions. We recognized that the overhead of storing this information would
potentially be quite high; in general, it requires storing information about
the lineage and intermediate partition of every intermediate
record. Nonetheless, we wanted to gain an understanding of the regimes in which
this overhead might be a reasonable tradeoff for a decreased recovery time. In
this section, we examine the potential benefits and disadvantages of this
approach using a microbenchmark.

To evaluate the potential time savings from selectively reading the subset of
input records needed to perform recovery, we created a 13.5 GB file on one of
our cluster's disks and compared the time taken to completely scan through the
file with the time taken to read a certain percentage of the file's
records. The percentage of records selectively read roughly corresponds to the
``selectivity'' of the recovery being simulated. For example, if 1\% of records
are being read, this corresponds to the amount of reading necessary to recover
from a lost of 1\% of the cluster's intermediate partitions.

As a simplifying assumption, we assumed that the records are evenly spaced
throughout the file. We completely purged the operating system's file buffer
cache and disabled any caching on our disk controllers so that each experiment
started from a cold cache.

As Figure~\ref{fig:seek-vs-scan} shows, when the selectivity of recovery is
quite small, selective reads can achieve large speedups over a sequential
scan. However, selectively reading records is far from proportional. For
example, for a file with 1KB records, the cost of sequentially scanning the
file is the same as the cost of selectively reading 1\% of its records; this
means that the loss of more than 1\% of the cluster's intermediate partitions
can be recovered from just as quickly by scanning input files as it can by
selectively reading them for I/O-bound jobs.

We suspect that this non-proportionality is due to a combination of the
overhead of seeking between records, the overhead of the relatively many
\texttt{read()} syscalls needed to retrieve those records, and the behavior of
the operating system's buffer cache. Note that for certain record sizes and
selectivities, selective reading performs dramatically worse than sequential
scanning; this is due mainly to poor interaction between the application and
the buffer cache.

In addition to its non-proportionality, selective reads have a negative impact
on the performance of other concurrent operations to the same
disk. Figure~\ref{fig:simultaneous_same_file} shows that selectively reading
from a file while scanning through it simultaneously can decrease the speed of
the scan by up to 600\%. We believe that cache interference between the two
writing processes, as well as the mechanical act of disrupting the sequentiality
of disk access with seeking, are the source of these overheads.

Figure~\ref{fig:simultaneous_different_files} shows the impact of running a
scan and a selective read over two different files on the same disk
simultaneously. Here the performance decrease for scans is much less drastic;
we believe the primary source of this performance decrease to be the overhead
imposed on the scan by disk seeks.

These results indicate, somewhat intuitively, that when the selectivity of the
recovery is very small (less than 0.01\%), it is highly beneficial to perform
selective reads. However, selective reads are only a proportional form of fault
tolerance if records are relatively large, and they have the potential to
interact poorly with other concurrent sequential scans. In addition, a recovery
with small selectivity is only likely when the cluster is fairly large, which
is a different operating environment from the ``dense'' clusters on which we
focus in this work.

\section{Evaluation}
\label{fault_tolerance:sec:eval}

In Section~\ref{fault_tolerance:sec:methodology}, we describe our experimental methodology. In
Section~\ref{sec:proportionality}, we show that recovery from both disk and
node failure are proportional, in that the time taken to recover is
proportional to the size of the failure. In
Section~\ref{sec:scan_sharing_overhead}, we show that the overhead imposed by
scan sharing a normal job with a recovery job is low.

\subsection{Methodology}
\label{fault_tolerance:sec:methodology}

We evaluated our fault tolerance mechanisms on eight of the machines in the
cluster described in Section~\ref{sec:hardware_architecture}. Each hard drive
is configured with a single XFS partition that is configured with a single
allocation group to avoid file fragmentation across allocations groups and is
mounted with the \texttt{noatime}, \texttt{nobarrier} and \texttt{noquota}
flags set. For this evaluation, all servers were running Linux 2.6.32. Jobs
source and sink data to HDFS, configured with 128MB blocks and whole-file
replication of the primary replica of each file.

\themis is written in C++ and, in this evaluation, is compiled with
\texttt{g++} 4.7.1. The cluster coordinator, node coordinator and HDFS
rewriting proxy are written in Python.

We rely on the sort MapReduce job to evaluate our fault tolerance
mechanisms. Since sort corresponds to no-op \map and \reduce functions, it
provides a natural way of evaluating fault tolerance independently of the job
being performed. At the same time, sort's large intermediate data set size
allowed us to stress-test the system's ability to scale
proportionally. Additionally, we have found it logistically difficult to both
obtain and store freely-available data sets that are sufficiently large that
they do not fit in a single node's memory. The input data set for sort is easy
to generate synthetically, which allows us to scale the evaluation beyond a
single node.

\subsection{Proportionality of Recovery}
\label{sec:proportionality}

In this section, we explore the proportionality of our recovery process in
response to disk and node failures.

\subsubsection{Recovering From Disk Failure}
\label{sec:disk_proportionality}

\begin{figure}
  \centering
  \includegraphics[width=\columnwidth]{fault_tolerance/graphs/disk_recovery_proportionality.pdf}
  \caption{\label{fig:disk_recovery_proportionality} Runtime of recovery from a
    disk failure during an 800GB sort with an increasing number of failed
    disks.}
\end{figure}

To test the proportionality of recovery from a disk failure, we ran an 800GB
sort job across our eight-node testbed and failed an increasing number of disks
during phase one. Disk failures were injected into the system by making the
part of \themis that writes to intermediate disks fail after it had written a
certain number of bytes to the disks we wanted to fail. We then ran a recovery
job to recover the data from those failed
disks. Figure~\ref{fig:disk_recovery_proportionality} shows the elapsed time of
both phases for the recovery job.

The elapsed time of the recovery job's phase two increases sub-linearly as the
number of disk failures increases.  This is because phase two is designed to
process multiple intermediate partitions in parallel and the number of
partitions created during recovery is fairly small, so processing twice as many
partitions doesn't necessarily take twice as much time. The decrease in phase
one's recovery time as the number of disks to recover increases is coincidental
and simply reflects the variability of access time provided by HDFS.

\subsubsection{Recovering From Node Failure}

\begin{figure}
  \centering
  \includegraphics[width=\columnwidth]{fault_tolerance/graphs/node_recovery_proportionality.pdf}
  \caption{\label{fig:node_recovery_proportionality} Runtime of recovery from
    node failures during an 800GB sort.}
\end{figure}

To test the proportionality of recovery from node failure, we ran the same
800GB sort job as in the disk failure tests, but instead of failing individual
disks, we killed all \themis-related processes on a set of nodes approximately
120 seconds after starting the job. Each DFS disk's input consists of ten
evenly-sized files, each approximately 1.6GB
long. Figure~\ref{fig:node_recovery_proportionality} shows the elapsed time of
both phases of the recovery jobs for these failures.

Phase one's recovery time increases drastically as the number of failures
increase. The primary reason for this is that the same amount of recovery must
be done across an increasingly small number of nodes. Recovery time in phase
one is further increased by the fact that nodes are typically not accessing the
whole-file-replicated primary replica of the files they are recovering; as a
result, read performance degrades to that of unmodified HDFS.

Phase two's recovery time increases sub-linearly for the same architectural
reasons that it increases sub-linearly during disk recovery. Due to end-of-file
acknowledgments, only a small number of duplicate records are generated. As a
result, phase two's node recovery is roughly equivalent to scan-sharing phase
two of a normal 800GB sort with a disk recovery for all of the failed nodes'
disks.

\begin{figure}[t]
  \centering
  \includegraphics[width=\columnwidth]{fault_tolerance/graphs/scan_share_overhead.pdf}
  \caption{\label{fig:scan_sharing_overhead} Comparing the baseline performance
    of an 800GB sort with the performance of scan-sharing that sort with disk
    failure recovery jobs.}
\end{figure}

\subsection{Scan Sharing Overhead}
\label{sec:scan_sharing_overhead}

To evaluate the overhead imposed on a normal job by scan-sharing it with a
recovery job, we ran an 800GB sort job scan-shared with the recovery jobs
described in Section~\ref{sec:disk_proportionality}. In
Figure~\ref{fig:scan_sharing_overhead}, we see that phase one's runtime remains
fairly flat until we scan-share the sort job recovery of eight disks. At this
point, the system is writing so much intermediate data to the remaining disks
that it transitions from being bound by the speed at which it can read from
HDFS to being bound by the speed at which it can write to its intermediate
disks. At the same time, the amount of intermediate data produced by the
recovery job becomes large enough to visibly impact phase two.

\section{Conclusions}

MapReduce's traditional approach to fault tolerance is proportional, but it
imposes the overhead of additional rounds of I/O in common-case operation,
which negatively impacts the system's overall performance. In this work, we
have shown that, through leveraging the multi-tenancy typical of a MapReduce
cluster and composing previously-known fault tolerance techniques, it is
possible to provide proportional fault tolerance without imposing additional
rounds of I/O in failure-free operation.

\section{Acknowledgments}

This work was sponsored in part by NSF Grants CSR-1116079 and MRI CNS-0923523,
and through donations by Cisco Systems and a NetApp Faculty Fellowship.

Chapter~\ref{chapter:fault_tolerance} contains material as it appears in the
Proceedings of the ACM Symposium on Cloud Computing (SOCC) 2012. Rasmussen,
Alexander; Porter, George; Vahdat, Amin. The dissertation author was the
primary investigator and author of this paper.

Chapter~\ref{chapter:fault_tolerance} contains material submitted for
publication as ``I/O-Efficient Fault Tolerance for MapReduce''. Rasmussen,
Alexander; Porter, George; Vahdat, Amin. The dissertation author was the
primary investigator and author of this paper.



\section{Conclusions}
\label{themis:sec:conclusions}

Many MapReduce jobs are I/O-bound, and so minimizing the number of I/O
operations is critical to improving their performance.  In this work, we
present Themis, a MapReduce implementation that meets the 2-IO property,
meaning that it issues the minimum number of I/O operations for jobs large
enough to exceed memory.  To avoid materializing intermediate results, Themis
foregoes task-level fault tolerance, relying instead on job-level fault
tolerance. Since the 2-IO property prohibits it from spilling records to disk,
Themis must manage memory dynamically and adaptively. To ensure that writes to
disk are large, Themis adopts a centralized, per-node disk scheduler that
batches records produced by different \mappers.

There exist a large and growing number of clusters that can process
petabyte-scale jobs, yet are small enough to experience a qualitatively lower
failure rate than warehouse-scale clusters.  We argue that these deployments
are ideal candidates to adopt more efficient implementations of MapReduce,
which result in higher overall performance than more pessimistic
implementations.  Themis has been able to implement a wide
variety of MapReduce jobs at nearly the sequential speed of the underlying
storage layer, and is on par with TritonSort's record sorting performance.


% Stuff at the end of the dissertation goes in the back matter
\backmatter
\bibliographystyle{plain} % Or whatever style you want like plainnat
\bibliography{references}

\end{document}
