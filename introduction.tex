\chapter{Introduction}
\label{chapter:introduction}

The quantity of data the world generates and stores is growing at a staggering
rate. Online retailers log users' purchase histories and interactions with
their websites in order to target advertising to their particular
interests. Walmart handles more than a million customer transactions per hour,
and the size of its customer database is estimated at 2.5
petabytes~\cite{economist-data-data-everywhere}. Search engines like Google
construct complex indices over the entire public Internet, which is estimated
to consist of at least 6.9 billion pages~\cite{worldwidewebsize}. Facebook's
users upload more than 300 million photos per
day~\cite{jay-parikh-slideshow}. Scientific instruments like the Australian
Square Kilometer Array, the Large Hadron Collider and the Pan-STARRS array of
telescopes can generate petabytes of data per day~\cite{fourth-paradigm}.

Capturing this data, while a technically challenging feat in and of itself, is
not enough. To be useful, this large volume of data must be analyzed,
aggregated, filtered, transformed and explored. This is no easy task. The
aforementioned data sets are but a few examples of a new class of ``big data''
-- data sets that are so large and complex that they become difficult to
process using traditional techniques and technologies.

Some data-intensive problems allow every record to be processed in parallel
without knowing anything about any other records. These problems are known as
``embarrassingly parallel'', and can be scaled out quite easily. For this class
of problems, simply splitting the data into small pieces and running the
desired computation over each piece is sufficient and can be scaled quite
easily. An example of an embarrassingly parallel problem is searching a corpus
of text for occurrences of a word; each document in the corpus can be scanned
independently and the results of the scan over each document can be trivially
merged together afterward.

However, a much larger class of problems are not embarrassingly parallel. These
problems require some form of aggregation or combination (analogous to group-by
and join in relational database systems) in addition to record-by-record
processing. Take, for example, the problem of counting the number of times each
word occurs in a corpus of text. The occurrences of each word can be counted in
each document independently, but these counts must then be added
together. Performing this aggregation efficiently is one of the primary
challenges facing designers of systems for computing on ``big data''.

\section{The Rise of Scale-Out Partition-Parallel Architectures}


In recent years, a range of large-scale, data-intensive systems have been
developed to tackle these kinds of workloads.  These systems work by
performing both record-by-record processing and aggregation in parallel whenever
possible by performing the computation over disjoint partitions of the
data.

One of the most popular frameworks for this form of analysis is
MapReduce~\cite{mapreduce}. A MapReduce computation is specified by two
functions. The first function, \map, takes a record as input and produces
zero or more records; it performs the record-by-record processing. Map
functions are generally assumed to be stateless and side-effect free so that
they are idempotent. The second function, \reduce, takes all records with
the same key as input and produces zero or more records. Colloquially, records
that have been passed through the \map function are said to have been
\emph{mapped}, while those that have been passed through the \reduce function
are said to have been \emph{reduced}.

MapReduce treats a data set as a collection of \emph{records}, each of which
consists of a \emph{key} and a \emph{value}. Both the key and the value can be
arbitrary. A record with key $k$ and value $v$ will be denoted \kvpair{k}{v}.

A canonical example MapReduce job is the problem of counting the occurrences of
each word in a text corpus. For this problem, the user might write a \map
function that takes a line of text as input and produces the record
\kvpair{word}{1} for each word in the line. The \reduce function would then
receive all records for a given word, add their values together, and produce a
single record \kvpair{word}{n}.

MapReduce's strength lies in the simplicity of its programming model. Users of
MapReduce need only write map and \reduce functions without concerning
themselves with dividing the data among nodes, performing inter-node
communication or handling node failure. The \map function's domain and its
idempotent nature make it embarrassingly parallel, while the \reduce function's
parallelism can be adjusted from completely serial to extremely parallel based
on the number of distinct keys in the records produced by the \map function.

MapReduce was developed by Google in the early 2000s for tasks like inverted
index generation and PageRank~\cite{pagerank} computation over Google's cache
of the web. Engineers at Yahoo! wrote an open-source version of MapReduce
called Hadoop~\cite{hadoop} in 2005 that has since become extremely popular and
is widely deployed in both academic and industrial settings.

\section{Scale-Out, but not Scale-Up}

While these systems scale quite well, they do not utilize their clusters'
resources to nearly the extent that they should. As one example, a cluster of
3452 nodes running Hadoop sorted 100 TB of data in 173
minutes~\cite{hadoop-sort-2009}. At a high level, this performance is quite
impressive -- an average of 578 GB of data sorted per minute. However, this
high-level performance masks a great deal of inefficiency. In the
aforementioned record-setting sort, each node in the cluster's average rate was
approximately 2.8 MBps, a small fraction of the speed at which that can read
from and write to its disks. This performance difference is even more shocking
when one considers that a significant fraction of the data (approximately 27
TB) could conceivably have been buffered in the cluster's main memory for
faster access.

These problems are not limited to Hadoop, however. Anderson and
Tucek~\cite{efficiency-matters} examined a collection of large-scale
data-intensive processing systems and found a widespread lack of efficiency
among them.

\begin{table}
\caption{\label{table:system-efficiency} Large scale sorting results over time,
  and their associated per-node and per-disk efficiency. Results extracted from
  ~\cite{efficiency-matters, hadoop-sort-2009, themis, tritonsort}.}
\centering
\begin{tabular}{|c|c|c|c|c|c|c|c|}
\hline
\textbf{Year} & \textbf{Name} & \textbf{Nodes} & \textbf{Disks} & \textbf{MB/s} & \textbf{MB/s/node} & \textbf{MB/s/disk}\\
\hline
2012 & Themis (35TB) & 20 & 320 & 4656 & 232.8 & 14.6 \\
2011 & Themis & 52 & 832 & 12080 & 232.4 & 14.5 \\
2011 & TritonSort & 52 & 832  & 15633 & 300.6 & 18.8 \\
2009 & Hadoop & 3452 & 13808 & 9633 & 2.79 & 0.69 \\
2009 & DEMSort & 195 & 780 & 9400 & 48.2 & 12.1\\
\hline
\end{tabular}
\end{table}

The tempting solution to the problem of low efficiency is to simply increase
the size of the cluster. Splitting the data being processed among progressively
more nodes decreases both the amount of data that each node must process and
(to the extent allowed by Amdahl's Law~\cite{amdahls_law}) increases the
throughput of the system. However, this approach has several negative
consequences.

Larger clusters have a proportionally large associated capital expense and
operational cost. Google, one of the pioneers of the current wave of largest
data-intensive systems, has contracted over 260MW to power its data
centers~\cite{google-dc-power-blog}. When it filed its IPO in 2011, Facebook
reported that it spent \$606 million on constructing and equipping its data
centers in 2011 and expected to spend another \$500 million in
2012~\cite{facebook-ipo}. As problem sizes increase, these expenses must also
increase.

Large data centers also have an environmental cost. McKinsey and Company
estimates that the carbon dioxide emissions from data centers will surpass
emissions from the airline industry by 2020~\cite{mckinsey-co2-study}.
Further, larger clusters are harder to manage and experience faults more
frequently than smaller clusters do because of the increased number of nodes in
those clusters. We will explore the implications of increased failure further
in Chapters~\ref{chapter:themis} and~\ref{chapter:fault_tolerance}.

\section{Sources of Inefficiency in Existing Systems}

While a thorough study of the sources of per-node inefficiency in existing
systems has not been performed, we can broadly classify three different sources
of inefficiency in systems that are I/O-bound:

\paragraph{Inefficient I/O} Current-generatione large scale data processing
systems read from and write to large collections of magnetic hard drives. These
magnetic drives are characterized by their fast sequential access and slow
random access. Fundamentally, systems that desire a high throughput from these
devices should write to them sequentially as much as possible.

\paragraph{Too much I/O per record} Existing
systems may read and write each record to disk several times before processing
is complete, either due to memory pressure or for fault tolerance. These
additional reads and writes incur significant additional overhead, as disk is
at least an order of magnitude slower than main memory or network transfer.

\paragraph{Imbalanced hardware configurations} Often, the hardware platforms on
which these systems are deployed are configured such that the system will run
out of network bandwidth or memory before they can maximize their disks'
throughput. This dissertation argues that a degree of software/hardware
co-design can lead to radically more efficient software and hardware
architectures.

\section{Hypothesis}

The hypothesis of this dissertation is that systems built with efficient disk
I/O as a first-order architectural concern can realize an order of magnitude
improvement in performance versus existing large-scale data-intensive systems
without compromising their scalability or generality.

In this dissertation, we argue that the chief challenges of building such a
system lie both in minimizing the number of I/O operations per record and in
ensuring that disk I/O is done sequentially as much as possible. We also argue
that significant increases in per-node efficiency can be realized by
considering alternative fault tolerance models to the task-level fault
tolerance schemes present in modern data-intensive systems.

We explore the design of radically more efficient data processing systems
through two main prototype systems: TritonSort, a large-scale sorting system,
and Themis, a large-scale MapReduce system.

\section{Organization}

TritonSort and Themis have each improved on the performance of systems in their
respective problem domains by almost an order of magnitude, approaching the
maximum throughput possible on the clusters on which they are
deployed. Further, we have demonstrated both analytically and experimentally
that a range of different fault tolerance schemes that incur less I/O than the
traditional task-based approach to fault tolerance are practical for a
wide-range of MapReduce deployments, and implemented a fault tolerance scheme
that does not impose additional rounds of disk I/O during failure-free
operation.

Chapter~\ref{chapter:background} provides background on the problem domains of
large-scale sorting and MapReduce. Chapter~\ref{chapter:principles} provides an
overview of the architecture and design principles that underpin the systems
presented in this dissertation. Chapter~\ref{chapter:related} explores related
work.  Chapter~\ref{chapter:tritonsort} presents the design and implementation
of TritonSort. Chapter~\ref{chapter:themis} presents the design and
implementation of Themis, focusing in particular on its differences from
TritonSort's design. Chapter~\ref{chapter:fault_tolerance} takes an in-depth
look at fault tolerance in Themis, looking first at the trade-off between fault
tolerance and I/O-efficiency, and then presenting the design and implementation
of an I/O-efficient fault tolerance scheme for Themis. The dissertation
concludes with Chapter~\ref{chapter:conclusions}, which describes some open
problems and future directions.
