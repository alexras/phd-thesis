\subsection{Responding to Failures}
\label{sec:fault_response}

As node coordinators run, they refresh a keep-alive key in Redis every few
seconds; if a node fails to refresh its keep-alive key, the cluster coordinator
presumes that the node has failed. A node notifies the cluster coordinator
directly if it finds that it can no longer write to one of its intermediate
disks.

\themis attempts to insulate the rest of the cluster from a failure whenever
one occurs so that the healthy portion of the cluster can complete as much work
as possible. To avoid the attendant complexity and fragility of coordinating
failure notification across nodes, \themis simply discards any data meant for a
failed portion of the cluster. When a node fails, all existing TCP sockets to
that node will break. Nodes respond to broken sockets by discarding all data
meant for that socket for the remainder of the batch. Similarly, when a disk
fails, all data that would have been written to the failed disk for the rest of
the batch is discarded. Subsequent batches will not use failed disks or nodes
until an operator has explicitly marked them as having recovered.

Currently, the user is responsible for issuing a recovery job to recover a
failed job. Scheduling recovery jobs to maximize the likelihood of scan sharing
is beyond the scope of this work; we examine some related efforts relevant to
this problem in Chapter~\ref{chapter:related_work}.
