% Put your maximum 350 word abstract here.
\begin{dissertationabstract}
As the volumes of data collected both by corporations and by
scientific institutions grows, there is a growing need for scalable,
data-intensive processing platforms to analyze and filter that data. The
effectiveness of these systems is measured by the quantity and quality of data
that they can process in a reasonable amount of time; thus, these systems have
very high I/O and storage requirements.

Existing systems are very effective at scaling to large cluster
sizes. Unfortunately, there exists a significant gap between the performance
these systems provide and the underlying capacity of the hardware
infrastructure on which they are deployed.

In this dissertation, I endeavor to bridge this performance gap. In particular,
I present two systems, TritonSort and Themis. TritonSort is a high-performance
large-scale sorting system capable of sorting 100TB of data on a modestly-sized
cluster at about 82\% of that cluster's peak hardware performance. Themis is an
extension of TritonSort that supports the popular MapReduce programming
paradigm and can run a wide spectrum of MapReduce jobs at nearly the speed at
which TritonSort can sort. I conclude with a discussion of the approach to
fault tolerance taken by TritonSort and Themis.

\end{dissertationabstract}
