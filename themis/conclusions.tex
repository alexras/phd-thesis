\section{Conclusions}
\label{themis:sec:conclusions}

Many MapReduce jobs are I/O-bound, and so minimizing the number of I/O
operations is critical to improving their performance.  In this work, we
present Themis, a MapReduce implementation that meets the 2-IO property,
meaning that it issues the minimum number of I/O operations for jobs large
enough to exceed memory.  To avoid materializing intermediate results, Themis
foregoes task-level fault tolerance, relying instead on job-level fault
tolerance. Since the 2-IO property prohibits it from spilling records to disk,
Themis must manage memory dynamically and adaptively. To ensure that writes to
disk are large, Themis adopts a centralized, per-node disk scheduler that
batches records produced by different \mappers.

There exist a large and growing number of clusters that can process
petabyte-scale jobs, yet are small enough to experience a qualitatively lower
failure rate than warehouse-scale clusters.  We argue that these deployments
are ideal candidates to adopt more efficient implementations of MapReduce,
which result in higher overall performance than more pessimistic
implementations.  Themis has been able to implement a wide
variety of MapReduce jobs at nearly the sequential speed of the underlying
storage layer, and is on par with TritonSort's record sorting performance.
